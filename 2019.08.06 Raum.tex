\documentclass{article}
\usepackage{amsmath,amsfonts,amsthm}
% \usepackage[german]{babel}
\usepackage[latin1]{inputenc}
% \usepackage[utf8]{inputenc}

\author{Thomas Hofmann}
\title{Das Wesen des Raumes}
\begin{document}
\maketitle

Als Lebewesen nehmen wir unsere Umgebung mit den Sinnen wahr und organisieren sie in einem �u�eren Raum. Mit Kant kann man sogar weiter gehen und den Raum als konstitutiv f�r den �u�eren Sinn �berhaupt, als reine Form sinnlicher Anschauung ansehen. 

\newpage


 dreidimensionalen Raum. Dieser geometrische (Euklidsche) Raum ist prinzipiell vermessbar 


und kann der Cartesischen Idee folgend mit einem Koordinatensystem versehen werden.



\end{document}