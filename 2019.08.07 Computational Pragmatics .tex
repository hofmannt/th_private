\documentclass{article}
\usepackage{amsmath,amsfonts,amsthm}
\author{Thomas Hofmann}
\title{Computational Pragmatics}
\begin{document}
\maketitle

\begin{enumerate}
\item Evolutionary game theory \cite{smith1982evolution} models interactions between living beings as a game with payoffs (increased fitness) determined by the collective behavior of populations. Importantly, the units of of natural selection are not primarily organisms, but heritable traits, i.e.~strategies (cf.~\cite{jager2007game}).
\item  Use of evolutionary game theory in cultural evolution requires to clarify the (micro-)mechnism of reproduction. Learning and imitation seem insufficient to capture rationality and creativity of humans, something best response dynamics aims to address \cite{jager2007game}. Here, agents are capable to design (optimal) strategies with regard to a given population and are not dependent on chance. 
\item Episitemic interpretation: bring small dose of rationality into a irrational population and see how it amplifies through increaing strategic depth \cite[Section 2.3]{jager2007game}.
\item Cautious deliberation leads from conventionalized semantics (what is said) to the pragmatic content (what is meant) \cite[Section 3]{jager2007game}. Communication can be modeled as a signalling game \cite{lewis2008convention,van2004signalling}: nature generates a possible world $W$, speaker strategy maps from $W \mapsto F$ to forms and hearer strategy maps back $F \mapsto W'$. 
\end{enumerate}

\bibliographystyle{acm}
\bibliography{pragmatics}

\end{document}