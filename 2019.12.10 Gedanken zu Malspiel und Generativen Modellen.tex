\documentclass[12pt,a4paper]{article}
\usepackage[top=30pt,bottom=30pt,left=48pt,right=46pt]{geometry}
\usepackage{amsmath,amssymb,amsthm,bm,bbm,hyperref}
\usepackage{titlesec}
\usepackage[latin1]{inputenc}


\setlength\itemsep{2mm}
\newcommand{\mb}{\mathbf}
\newcommand{\mc}{\mathcal}
\newcommand{\ZZ}{\mathcal Z}
\renewcommand{\Re}{\mathbb R}
\newcommand{\E}{\mathbf E}
\newcommand{\rade}{\mathcal R}
\newcommand{\funct}{\mathcal F}
\newcommand{\llangle}{\left \langle}
\newcommand{\rrangle}{\right \rangle}
\newtheorem{theorem}{Theorem}
\newtheorem{axiom}{Axiom}
\newtheorem{definition}{Definition}
\newtheorem{lemma}{Lemma}
\newtheorem{corollary}{Corollary}
\newtheorem{example}{Example}
\newtheorem{proposition}{Proposition}


\titleformat{\section}
  {\normalfont\large\bfseries}
  {\thesection}{1em}{}
\titleformat{\subsection}
  {\normalfont\normalsize\bfseries}
  {\thesubsection}{1em}{}
  
  
\author{Thomas Hofmann}
\title{Gedanken zu Malspiel und Generativen Modellen}
\begin{document}
\maketitle

\section{Schrift im Schreiben und Malen}

\subsection{Begegung mit Buchstaben im Spiel}

\paragraph{Arno Stern}
\textit{Es entsteht eine \"Ahnlichkeit zwischen den Buchstaben, die es wahrnimmt und den Erstfiguren. [...] Und ebenso, wie sich alle Elemente der Formulation bei jedem Kind bilden, werden die Buchstaben einer nach dem anderen entdeckt.}\\

Stern konzentriert sich auf Gro�buchstaben des lateinischen Alphabets, das sich im Vergleich zu vielen anderen Alphabeten durch gro�e geometrische Einfachheit auszeichnet. Er konstatiert, dass dem Kind diese Buchstaben zuerst als Einzelbuchstaben auffallen, also (noch) nicht im Verbund von Silben oder W�rtern. Dabei ist ihm die graphische Komposition aus elementaren Strichen und Kreiselementen wichtig. Methodisch nimmt er teilweise Schreibvariationen der Buchstaben als  Hinweis f�r die Identifizierung mit Erstfiguren: etwa wenn ein \textbf{A} eher wie ein Haus aussieht, der \textbf{R} eher wie ein Mensch und das \textbf{E} viele Fransen angeheftet bekommt.

Was das Auftreten der Buchstaben betrifft, so ist das fl�chenf�llende Muster typisch, also eine texturierte Anordnung mit gewissen Regularit�ten (Gr��e, Orientierung, Dichte). Andere h�ufig aufretende Bilder verwenden Buchstaben in zuf�lligen Anordnungen, h�ufig auch gespiegelt oder auf dem Kopf stehend. Im Schreibspiel, wo die Schrift naturgem�� dominiert, treten oft imaginierte Schreiblinien als strukturgebende Elemente hinzu, auf denen sich die Buchstaben anordnen. Solche horizontale Kompositionen finden sich selbst da, wo Schrift eher als verbundene Schreibschrift -- quasi als Gekrakel -- zu Papier gebracht wird.  Die Schrift kann im Malspiel aber auch, insbesondere wenn sich Buchstaben zu W�rtern f�gen, Teil einer Inszenierung sein, etwa als Annotation und Beschriftung von Personen oder Gegenst�nden.

\subsection{Generative Modelle f�r Schriftzeichen}

\paragraph{Chinesische Zeichen mit Stil} Versuch der Transformation von Standardschriftzeichen in die Handschrift eines Kalligraphen \cite{chang2018generating}.



\bibliography{th}
\bibliographystyle{acm}

\end{document}
