\documentclass[compress]{beamer}
\usetheme{Copenhagen}
% \setbeamertemplate{section in toc}[sections numbered]
\usepackage{dl2019}
\usepackage{hyperref}

%\usepackage{verbatim}
%\usepackage{algorithmic}
%\usepackage{mathtools}
%\usepackage{xcolor, graphicx}
\graphicspath{{./figures/}}

\title{
%%%%%%%%%%%%%%%%%%%%
Machine Learning Seminar \\ 252-4811-00L 
%%%%%%%%%%%%%%%%%%%%
}
\date{25 September 2019}
\author{Thomas Hofmann \& Gunnar R\"atsch, ETH Zurich}

\begin{document}
\maketitle 

%%%%%%%%%%%%%%%%%%%%
\begin{frame}
\frametitle{Goal of the Seminar}

\begin{enumerate} \is{8mm}
\item Practice giving scientific presentations
\item Critical assessment of research papers; discussion
\item Learn about advanced topics in machine learning
\end{enumerate}

\end{frame}
%%%%%%%%%%%%%%%%%%%%
\begin{frame}
\frametitle{Organization \& Structure}

\begin{enumerate} \is{8mm}
\item Block seminar, 2x Saturdays: November 16 \& 23, HG E21
\item Participation on both days mandatory
\item Structure: 9:00-10:30, 11:00-12:30, 13:30-15:00, 15:30-17:00\\[2mm]
Maximum of 4x 1.5h blocks, 3x talks per block, 24 talks total
\item 20' + 10' = 30' per talk 
\end{enumerate}
\end{frame}

%%%%%%%%%%%%%%%%%%%%
\begin{frame}
\frametitle{Grading Criteria}

\begin{enumerate} \is{3mm}
\item Structure (how well is your talk organized?)
\item Understandability (how understandable is your oral presentation and slide design?)
\item  Completeness (how well do you provide right background, and manage to focus on what is important and relevant?)
\item  Activity (how engaged are you in class and in the talk preparation?)
\item Independence (how independent are you in preparing the presentation, and in reflecting on the paper?)
\end{enumerate} 

\end{frame}

%%%%%%%%%%%%%%%%%%%%
\begin{frame}
\frametitle{Presentation: Expectations}

\begin{enumerate} \is{3mm}
\item Use electronic slides (ppt, pdf, ...)
\item Talk length: 20 min + 10 min discussion
\item  Provide sufficient background to be understandable\\ (for someone who has taken an ML class)
\item Present contributions and results; criticial reflection
\end{enumerate} 

\end{frame}

%%%%%%%%%%%%%%%%%%%%
\begin{frame}
\frametitle{Presentation: Typical Structure}
\begin{enumerate} \is{3mm}
\item Introduction (Motivation and background)
\item Formal problem statement (Notation, ...)
\item Technical contribution (algorithm, theoretical result, ...)
\item Experimental results (if any)
\item Discussion (what are perceived strengths and weaknesses of the paper; what could be done more; ...)
\item Conclusion
\end{enumerate} 
\end{frame}

%%%%%%%%%%%%%%%%%%%%
\begin{frame}
\frametitle{Presentation: Recommendations}
\begin{enumerate} \is{4mm}
\item Giving compelling presentations is hard: Minimize words and maximize illustrations!
\item Focus on giving intuition, identifying key insights etc.
\item Some pieces of advice:\\[2mm]
\begin{itemize}\is{2mm}
\item \url{http://www.cs.berkeley.edu/~jrs/speaking.html}
\item \url{http://research.microsoft.com/en- us/um/people/simonpj/papers/giving-a-talk/giving-a- talk.htm}
\item \url{http://greatresearch.org/2013/10/04/presenting-a- technical-talk/}
\end{itemize} 
\end{enumerate} 
\end{frame}

%%%%%%%%%%%%%%%%%%%%
\begin{frame}
\frametitle{Goal for Today}
\begin{enumerate} \is{4mm}
\item Short description of each paper 
\item Choice of top 4 preferences -- \textbred{issued via \href{}{Google form}}
\item Take note of paper numbers, checkmark in form, submit 
\item We will try to perform a good matching  in the next days...
\item Additional iterations via email, if necessary
\end{enumerate} 
\end{frame}

%%%%%%%%%%%%%%%%%%%%
\begin{frame}
\frametitle{Paper \#1}
Shalev-Shwartz, S., Singer, Y., Srebro, N. and Cotter, A., 2011. Pegasos: Primal estimated sub-gradient solver for svm. Mathematical programming, 127(1), pp.3-30.\\[3mm]
We describe and analyze a simple and effective stochastic sub-gradient descent algorithm for solving the optimization problem cast by Support Vector Machines (SVM). [..]
\end{frame}
%%%%%%%%%%%%%%%%%%%%
\begin{frame}
\frametitle{Paper \#2}
Duchi, J., Hazan, E. and Singer, Y., 2011. Adaptive subgradient methods for online learning and stochastic optimization. Journal of Machine Learning Research, 12(Jul), pp.2121-2159.\\[3mm]
We present a new family of subgradient methods that dynamically incorporate knowledge of
the geometry of the data observed in earlier iterations to perform more informative gradient-
based learning. [...]
\end{frame}
%%%%%%%%%%%%%%%%%%%%
\begin{frame}
\frametitle{Paper \#3}
Kingma, D.P. and Ba, J., 2014. Adam: A method for stochastic optimization. arXiv preprint arXiv:1412.6980.\\[3mm]
We introduce Adam, an algorithm for first-order gradient-based optimization of stochastic
objective functions, based on adaptive estimates of lower-order moments. The method is
straightforward to implement, is computationally efficient, has little memory requirements, is
invariant to diagonal rescaling of the gradients, and is well suited for problems that are large
in terms of data and/or parameters.
[...]
\end{frame}
%%%%%%%%%%%%%%%%%%%%
\begin{frame}
\frametitle{Paper \#4}
LeCun, Y.A., Bottou, L., Orr, G.B. and M�ller, K.R., 2012. Efficient backprop. In Neural networks: Tricks of the trade (pp. 9-48). Springer, Berlin, Heidelberg.\\[3mm]
The convergence of back-propagation learning is analyzed so as to explain common
phenomenon observed by practitioners. Many undesirable behaviors of backprop can be
avoided with tricks that are rarely exposed in serious technical publications. This paper
gives some of those tricks, and offers explanations of why they work.
[...]
\end{frame}
%%%%%%%%%%%%%%%%%%%%
\begin{frame}
\frametitle{Paper \#5}
Belkin, M. and Niyogi, P., 2003. Laplacian eigenmaps for dimensionality reduction and data representation. Neural computation, 15(6), pp.1373-1396.\\[3mm]
One of the central problems in machine learning and pattern recognition is to develop
appropriate representations for complex data. We consider the problem of constructing a
representation for data lying on a low-dimensional manifold embedded in a high-
dimensional space.[...]
\end{frame}
%%%%%%%%%%%%%%%%%%%%
\begin{frame}
\frametitle{Paper \#6}
Blei, D.M., Ng, A.Y. and Jordan, M.I., 2003. Latent dirichlet allocation. Journal of machine Learning research, 3(Jan), pp.993-1022.\\[3mm]
We describe latent Dirichlet allocation (LDA), a generative probabilistic model for collections
of discrete data such as text corpora. LDA is a three-level hierarchical Bayesian model, in
which each item of a collection is modeled as a finite mixture over an underlying set of
topics. Each topic is, in turn, modeled as an infinite mixture over an underlying set of topic
probabilities. [...]
\end{frame}
%%%%%%%%%%%%%%%%%%%%
\begin{frame}
\frametitle{Paper \#7}
Extracting and composing robust features with denoising autoencoders\\[3mm]
Previous work has shown that the difficulties in learning deep generative or discriminative
models can be overcome by an initial unsupervised learning step that maps inputs to useful
intermediate representations. We introduce and motivate a new training principle for
unsupervised learning of a representation based on the idea of making the learned
representations robust to partial corruption of the input pattern.  [...]
\end{frame}
%%%%%%%%%%%%%%%%%%%%
\begin{frame}
\frametitle{Paper \#8}
UMAP: Uniform Manifold Approximation and Projection for Dimension Reduction\\[3mm]
UMAP (Uniform Manifold Approximation and Projection) is a novel manifold learning
technique for dimension reduction. UMAP is constructed from a theoretical framework based
in Riemannian geometry and algebraic topology. The result is a practical scalable algorithm
that applies to real world data.  [...]
\end{frame}
%%%%%%%%%%%%%%%%%%%%
\begin{frame}
\frametitle{Paper \#9}
Disentangling by Factorizing \\[3mm]
We define and address the problem of unsupervised learning of disentangled representations on data generated from independent factors of variation. We propose FactorVAE, a method that disentangles by encouraging the distribution of representations to be factorial and hence independent across the dimensions. We show that it improves upon ?-VAE by providing a better trade-off between disentanglement and reconstruction quality. [...]
\end{frame}
%%%%%%%%%%%%%%%%%%%%
\begin{frame}
\frametitle{Paper \#10}
Random Features for Large-Scale Kernel Machines\\[3mm]
To accelerate the training of kernel machines, we propose to map the input data
to a randomized low-dimensional feature space and then apply existing fast linear
methods. The features are designed so that the inner products of the transformed
data are approximately equal to those in the feature space of a user specified shiftinvariant kernel. We explore two sets of random features, provide convergence
bounds on their ability to approximate various radial basis kernels, and show
that in large-scale classification and regression tasks linear machine learning algorithms applied to these features outperform state-of-the-art large-scale kernel
machines.
\end{frame}
%%%%%%%%%%%%%%%%%%%%
\begin{frame}
\frametitle{Paper \#11}
Adversarial feature learning\\[3mm]
[...] GANs have no means of learning the inverse mapping -- projecting data back into the
latent space. We propose Bidirectional Generative Adversarial Networks (BiGANs)
as a means of learning this inverse mapping, and demonstrate that the resulting
learned feature representation is useful for auxiliary supervised discrimination tasks,
competitive with contemporary approaches to unsupervised and self-supervised
feature learning. [...]
\end{frame}
%%%%%%%%%%%%%%%%%%%%
\begin{frame}
\frametitle{Paper \#12}
Chen, T.Q., Rubanova, Y., Bettencourt, J. and Duvenaud, D.K., 2018. Neural ordinary differential equations. In Advances in neural information processing systems (pp. 6571-6583).\\[3mm]
We introduce a new family of deep neural network models. Instead of specifying a discrete
sequence of hidden layers, we parameterize the derivative of the hidden state using a
neural network. The output of the network is computed using a blackbox differential equation
solver. These continuous-depth models have constant memory cost, adapt their evaluation
strategy to each input, and can explicitly trade numerical precision for speed. [...]
\end{frame}
%%%%%%%%%%%%%%%%%%%%
\begin{frame}
Kipf, T.N. and Welling, M., 2016. Semi-supervised classification with graph convolutional networks. arXiv preprint arXiv:1609.02907.\frametitle{Paper \#13}
\\[3mm]
We present a scalable approach for semi-supervised learning on graph-structured data that
is based on an efficient variant of convolutional neural networks which operate directly on
graphs. We motivate the choice of our convolutional architecture via a localized first-order
approximation of spectral graph convolutions. [...]
\end{frame}
%%%%%%%%%%%%%%%%%%%%
\begin{frame}
\frametitle{Paper \#14}
Generative Adversarial Networks\\[3mm]
We propose a new framework for estimating generative models via an adversarial process, in which we simultaneously train two models: a generative model G
that captures the data distribution, and a discriminative model D that estimates
the probability that a sample came from the training data rather than G. The training procedure for G is to maximize the probability of D making a mistake. This
framework corresponds to a minimax two-player game.
 [...]
\end{frame}
%%%%%%%%%%%%%%%%%%%%
\begin{frame}
\frametitle{Paper \#15}
Attention Is All You Need\\[3mm]
The dominant sequence transduction models are based on complex recurrent or
convolutional neural networks that include an encoder and a decoder. The best
performing models also connect the encoder and decoder through an attention
mechanism. We propose a new simple network architecture, the Transformer,
based solely on attention mechanisms, dispensing with recurrence and convolutions
entirely.  [...]
\end{frame}
%%%%%%%%%%%%%%%%%%%%
\begin{frame}
\frametitle{Paper \#16}
Deep Residual Learning for Image Recognition\\[3mm]
Deeper neural networks are more difficult to train. We
present a residual learning framework to ease the training
of networks that are substantially deeper than those used
previously. We explicitly reformulate the layers as learning residual functions with reference to the layer inputs, instead of learning unreferenced functions.  [...]
\end{frame}
%%%%%%%%%%%%%%%%%%%%
\begin{frame}
\frametitle{Paper \#17}
Schulman, J., Levine, S., Abbeel, P., Jordan, M. and Moritz, P., 2015, June. Trust region policy optimization. InInternational conference on machine learning(pp. 1889-1897).\\[3mm]
In this article, we describe a method for optimizing control policies, with guaranteed monotonic
improvement. By making several approximations to the theoretically-justified scheme, we develop a practical algorithm, called Trust Region
Policy Optimization (TRPO). T [...]
\end{frame}
%%%%%%%%%%%%%%%%%%%%
\begin{frame}
\frametitle{Paper \#18}
Lillicrap, T.P., Hunt, J.J., Pritzel, A., Heess, N., Erez, T., Tassa, Y., Silver, D. and Wierstra, D., 2015. Continuous control with deep reinforcement learning. arXiv preprint arXiv:1509.02971.\\[3mm]
We adapt the ideas underlying the success of Deep Q-Learning to the continuous action
domain. We present an actor-critic, model-free algorithm based on the deterministic policy
gradient that can operate over continuous action spaces. Using the same learning algorithm,
network architecture and hyper-parameters, our algorithm robustly solves more than 20
simulated physics tasks, including classic problems such as cartpole swing-up, dexterous
manipulation, legged locomotion and car driving. [...]
\end{frame}
%%%%%%%%%%%%%%%%%%%%
\begin{frame}
\frametitle{Paper \#19}
Tamar, A., Wu, Y., Thomas, G., Levine, S. and Abbeel, P., 2016. Value iteration networks. In Advances in Neural Information Processing Systems (pp. 2154-2162).\\[3mm]
We introduce the value iteration network (VIN): a fully differentiable neural network with
aplanning module'embedded within. VINs can learn to plan, and are suitable for predicting
outcomes that involve planning-based reasoning, such as policies for reinforcement
learning. Key to our approach is a novel differentiable approximation of the value-iteration
algorithm, which can be represented as a convolutional neural network, and trained end-to-
end using standard backpropagation.  [...]
\end{frame}
%%%%%%%%%%%%%%%%%%%%
\begin{frame}
\frametitle{Paper \#20}
Learning from demonstrations\\[3mm]
Consider learning a policy from example expert behavior, without interaction with the expert or access to reinforcement signal. One approach is to recover the expert's cost function with inverse reinforcement learning, then extract a policy from that cost function with reinforcement learning. This approach is indirect and can be slow. We propose a new general framework for directly extracting a policy from data, as if it were obtained by reinforcement learning following inverse reinforcement learning [...]
\end{frame}
%%%%%%%%%%%%%%%%%%%%
\begin{frame}
\frametitle{Paper \#21}
Kingma, D.P. and Welling, M., 2013. Auto-encoding variational bayes. ICLR 2014.\\[3mm]
How can we perform efficient inference and learning in directed probabilistic models, in the
presence of continuous latent variables with intractable posterior distributions, and large
datasets? We introduce a stochastic variational inference and learning algorithm that scales
to large datasets and, under some mild differentiability conditions, even works in the
intractable case [...]
\end{frame}
%%%%%%%%%%%%%%%%%%%%
\begin{frame}
\frametitle{Paper \#22}
Tutorial on Variational Autoencoders\\[3mm]
In just three years, Variational Autoencoders (VAEs) have emerged
as one of the most popular approaches to unsupervised learning of
complicated distributions. VAEs are appealing because they are built
on top of standard function approximators (neural networks), and
can be trained with stochastic gradient descent. [...] This tutorial introduces the
intuitions behind VAEs, explains the mathematics behind them, and
describes some empirical behavior. No prior knowledge of variational
Bayesian methods is assumed.
\end{frame}
%%%%%%%%%%%%%%%%%%%%
\begin{frame}
\frametitle{Paper \#23}
Namkoong, H. and Duchi, J.C., 2017. Variance-based regularization with convex objectives. In Advances in Neural Information Processing Systems (pp. 2971-2980).\\[3mm]
We develop an approach to risk minimization and stochastic optimization that provides a convex surrogate for variance, allowing near-optimal and computationally
efficient trading between approximation and estimation error. Our approach builds
off of techniques for distributionally robust optimization and Owen?s empirical
likelihood, and we provide a number of finite-sample and asymptotic results characterizing the theoretical performance of the estimator [...]
\end{frame}
%%%%%%%%%%%%%%%%%%%%
\begin{frame}
\frametitle{Paper \#24}
Understanding deep learning requires rethinking generalization\\[3mm]
Despite their massive size, successful deep artificial neural networks can exhibit a
remarkably small difference between training and test performance. Conventional wisdom
attributes small generalization error either to properties of the model family, or to the
regularization techniques used during training.  [...]
\end{frame}
%%%%%%%%%%%%%%%%%%%%
\begin{frame}
\frametitle{Paper \#25}
Bahdanau, D., Cho, K. and Bengio, Y., 2014. Neural machine translation by jointly learning to align and translate. arXiv preprint arXiv:1409.0473.\\[3mm]
Neural machine translation is a recently proposed approach to machine translation. Unlike
the traditional statistical machine translation, the neural machine translation aims at building
a single neural network that can be jointly tuned to maximize the translation performance.
The models proposed recently for neural machine translation often belong to a family of
encoder-decoders and consists of an encoder that encodes a source sentence into a fixed-
length vector from which a decoder generates a translation. [...]
\end{frame}
%%%%%%%%%%%%%%%%%%%%
\begin{frame}
\frametitle{Paper \#26}
Natural Language Processing (Almost) from Scratch\\[3mm]
We propose a unified neural network architecture and learning algorithm that can be applied
to various natural language processing tasks including part-of-speech tagging, chunking,
named entity recognition, and semantic role labeling. This versatility is achieved by trying to
avoid task-specific engineering and therefore disregarding a lot of prior knowledge. Instead
of exploiting man-made input features carefully optimized for each task, our system learns
internal representations on the basis of vast amounts of mostly unlabeled training data.  [...]
\end{frame}
%%%%%%%%%%%%%%%%%%%%
\begin{frame}
\frametitle{Paper \#27}
Do ImageNet Classifiers Generalize to ImageNet?\\[3mm]
We build new test sets for the CIFAR-10 and ImageNet datasets. Both benchmarks have
been the focus of intense research for almost a decade, raising the danger of overfitting to
excessively re-used test sets. By closely following the original dataset creation processes,
we test to what extent current classification models generalize to new data. [...]
\end{frame}
%%%%%%%%%%%%%%%%%%%%
\begin{frame}
\frametitle{Paper \#28}
Support-vector networks\\[3mm]
The support-vector network is a new leaming machine for two-group classification problems. The
machine conceptually implements the following idea: input vectors are non-linearly mapped to a very highdimension feature space. In this feature space a linear decision surface is constructed. Special properties of the
decision surface ensures high generalization ability of the learning machine. The idea behind the support-vector
network was previously implemented for the restricted case where the training data can be separated without
errors. We here extend this result to non-separable training data.  [...]
\end{frame}

\end{document}
