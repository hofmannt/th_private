\documentclass[12pt,a4paper]{article}
\usepackage[top=50pt,bottom=50pt,left=48pt,right=46pt]{geometry}
\usepackage{amsmath,amssymb,amsthm,bm,bbm,hyperref}
\usepackage{titlesec}
\usepackage{graphicx}
\graphicspath{{./images/}}

\newcommand{\rPhi}{{\underline{\Phi}}}
\setlength\itemsep{2mm}
\newcommand{\mb}{\mathbf}
\newcommand{\mc}{\mathcal}
\newcommand{\x}{{\mathbf x}}
\newcommand{\ZZ}{\mathcal Z}
\renewcommand{\Re}{\mathbb R}
\newcommand{\E}{\mathbf E}
\newcommand{\rade}{\mathcal R}
\newcommand{\funct}{\mathcal F}
\newcommand{\llangle}{\left \langle}
\newcommand{\rrangle}{\right \rangle}
\newtheorem{theorem}{Theorem}
\newtheorem{axiom}{Axiom}
\newtheorem{definition}{Definition}
\newtheorem{lemma}{Lemma}
\newtheorem{corollary}{Corollary}
\newtheorem{example}{Example}
\newtheorem{detail}{Detail}
\newtheorem{proposition}{Proposition}
\newtheorem{comment}{Comment}


\titleformat{\section}
  {\normalfont\large\bfseries}
  {\thesection}{1em}{}
\titleformat{\subsection}
  {\normalfont\normalsize\bfseries}
  {\thesubsection}{1em}{}
  
  

\author{Thomas Hofmann}
\title{Weinberg: Quantum Field Theory.}
\date{\today}

\begin{document}

\maketitle

\section{History}

\section{Relativistiv Quantum Mechanics}

\begin{itemize}
\item \textit{Physical states} are rays in a complex-valued Hilbert space. They are normalized $\langle \Phi, \Phi \rangle = 1$ and one has to identify elements $\Phi, \Phi'$ whenever $\Phi' = \alpha \Phi$ with $\alpha \in \mathbb  C$, where $|\alpha|=1$.
\item \textit{Observables} are represented as Hermitain operators: $\Phi \mapsto A \Phi$ that are self-adjoint $A = A^\dagger$. They form a real vector space and generalize scalar, real-valued measurements. A state $\Phi$ has a definite value with regard to $A$, if it is an eigenvector of $A$, in which case $A\Phi= \lambda \Phi$, $\lambda \in \Re$.   Eigenvectors with different $\lambda$ are orthogonal. 
\item  \textit{Born rule}. Assume a ray $\rPhi$ is subjected to measurements that aim to determine whether is is in any state $\rPhi_1,\dots,\rPhi_n$, then the probabilit to find it in state $\rPhi_i$ is given by 
\begin{align}
P(\rPhi \to \rPhi_i) =  |\langle \Phi, \Phi_i \rangle|^2 
\end{align}
\item \textit{Wigner's theorem}. We may consider transformations of representations of states $\rPhi \to \rPhi'$, e.g.~for different observers. However, we require that observables do not change in the sense that for $\Phi' = T \Phi$ 
\begin{align}
P(\rPhi \to \rPhi_i) = P(\rPhi' \to \rPhi'_i)
\end{align}
This is equivalent to transformations obeying $\langle \Phi, \Psi \rangle = \langle T \Phi, T \Psi \rangle$. Wigner's theorem states that either $T$ is linear and unitary or anti-linear and anti-unitary. As the identity operator is linear and unitary, the linear and unit ray operators are more important in physics as they can often be thought of in an infinitesimal limit.
\item \textit{Group structure}. Invariance transformation have group structure. However operating on vectors (nor rays) means:  phase may change, yet this change only depends on the operator. This leads to projective group representations. However it is shown later that groups can be enlarged to obtain non-projective representations without phase changes. [\textit{Requires diving deeper into Lie group representations.}]
\item \textit{Connected Lie group}. Finite dimensional parameterization, each operator is continuously connected with the identity, $T(0) = I$, $T(\theta_1) T(\theta_2) = T(f(\theta_1,\theta_2))$, where $f^a(0,\theta) = f^a(\theta,0) = \theta^a$. Operators can be represented by power series (in finite neighborhood)
\begin{align}
U(T(\theta)) = 1 +  i \theta^a t_a + \frac 12 \theta^b \theta^c  t_{bc} + \dots
\end{align}
with conditions: (1) $t_a$ is Hermitian, (2) $t_{bc}$ is symmetric. If $U(T(\theta))$ is a non-projective representation of the Lie group, then one can derive a (necessary) consistency condition that reads $t_{bc} = - t_b t_c - i f^a_{bc} t_a$, with $f^a(\theta',\theta) = \theta+ \theta' + f^a_{bc} \theta_b' \theta_c+ ...$. 
\item \textit{Lie algebra}. Together with the symmetry requirements, one finally gets a set of commutation relations of a Lie algebra  
\begin{align}
[t_b, t_c]  = C^a_{bc} t_a, \quad C^a_{bc} = -f^a_{bc} + f^a_{cb} 
\end{align}
\item \textit{Abelian group}. If $f^a(\bar \theta,\theta) =\bar \theta^a + \theta^a$ then $[t_b, t_c] =0$ $\forall b,c$. Then one finds 
\begin{align}
U(T(\theta)) = \exp[i t_a \theta^a]
\end{align}
\item \textit{Special relativity.} Lorentz transform invariance
\begin{align}
d\mb x^2 - dt^2 = d \mb x'^2 - dt'^2
\end{align}
This is motivated from speed of light requirement $| d\mb x / dt |= 1$, hence $d \mb x^2 - dt^2= 0$. Lorentz transformations are affine $x'^\mu = \Lambda^\mu_\nu x^\nu + a^\mu$ with conditions as follows
\begin{align}
\eta_{\mu \nu} = \begin{cases} -1, & \mu=\nu=0 \\ 1 & \mu=\nu \in \{1,2,3\} \\ 0 & \text{otherwise} \end{cases}, \quad 
\eta_{\mu \nu} \Lambda^\mu_\rho \Lambda^\nu_\sigma = \eta_{\rho \sigma} 
\quad \text{or} \quad \eta^{\mu \nu} = \Lambda^\mu_\rho \Lambda_\sigma^\nu \eta^{\rho \sigma}
\end{align}
\item \textit{Inhomegenous Lorentz group}. Lorentz transformations form a group with composition
\begin{align}
T(\Lambda',a') T(\Lambda,a) = T(\Lambda' \Lambda, \Lambda'a + a') \,.
\end{align}
The transformations are unitary as $\text{det}(\Lambda)^2 =1$. This is also know as the Poincar\'e group.
\item \textit{Proper orthochronous Lorentz group}: Subgroup with $\text{det}(\Lambda)=1$ and $\Lambda^0_0 \ge 1$. The whole Lorentz group can be obtained from the poLg via space  inversion and time reversal operations.
\item \textit{Poincar\'e algebra}. 
\end{itemize}

\end{document}
