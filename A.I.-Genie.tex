\documentclass[a4]{article}

\newcommand{\slide}[1]{{[SLIDE: {#1}]}}

\author{Thomas Hofmann}
\title{A.I. -- Genie in a Bottle?}
\begin{document}

\maketitle

\section{Quest for Intelligence}

Intelligence is one of the most valued human skills. Our success as individuals as well as a civilization depends on it. Hence, the emergence of intelligent machines will greatly alter our lives and value systems. What then is intelligence? At a high level it is  the ability to understand, to make sense of the non-sensical, and to act accordingly. As such, intelligence is multi-faceted. It may involve different types of information processing such as recognition, concept formation, learning, inference, reasoning, and planning. We often think of these as cognitive abilities. However, intelligence is already built into the way we perceive and interprete the world through our senses. \\

\noindent We associate intelligence with ingenuity, a certain depth of thought, the ability to go beyond the surface. Yet, it also depends on experience (i.e.~breadth of information and data available) and quickness (i.e.~speed of information processing). The intelligence of our species has been advanced by the process of evolution over millions of years. Our formation and education as individuals takes a good part of our lifetime. However, nothing indicates that we have reached the end of intelligence. It is obvious and part of our human condition that we are inherently limited. Thus we have invented tools of information processing to complement our weaknesses and more and more of these have developed an intelligence of their own.\\

\noindent Automata have been a topos of imagination for centuries and reflections on intelligent machines were fueled in the 1950ies with the birth of the modern computer. In several waves, A.I.~research has emerged and re-emerged, but it has failed to leave a noticeable impact except in niches.\footnote{It has to be taken into account, that we may be blind towards new forms of intelligence. For instance, is a search engine `intelligent'?}  However, with large data sets (big data), unfathomable compute power (data centers, distributed computing systems), and innovative methodology (deep learning), a force has built up that is and will be changing much about everything we do. It is as if we have unleased a Genie that is not going to squeeze back into the bottle. Not so much because the Genie has a will of its own, but rather because we cannot effort to do so for the sake of progress.\\

\noindent A.I.~will challenge our intelligence in different ways, perhaps in different ways than we think. First, it will get better at what we are good at. Inevitably, it will not stop at our level, but advance beyond it. Certain types of A.I.~may not exist (yet), but once they do and reach human levels, they will rapidly evolve beyond.\footnote{I am not talking about 'strong' A.I.~here, but this is a law for all specific forms of human-like intelligence.} This law is amplified by the fact that A.I.~improves, once deployed. The ability to learn, adapt and improve itself in operation is in its DNA. Second, there will be new forms of machine intelligence that only relate to human intelligence by analogy. We cannot read the genetic code of life. Future machines will. It is like we are lacking certain senses. Also we cannot optimally operate processes as complex as traffic, commodity chains or power grids at a national or global level, machines can 24/7. We are fundamentally bound by attention and as such we cannot attend to everything at the same time. Third, our collective intelligence is based on how we organize ourselves -- communication and coordination is key. Information technology has helped us with that, but imagine how an interconnected A.I.~will look like. A.I.~may not have a body or a face\footnote{The movie \textit{Her}.} and it may not need a soul. A.I.~has an uncomfortable air of invisibility. We may individualize A.I.~for human comfort, but this is not more than a surface presentation. 

\section{Sense from Senses}

One form of intelligence is strategic thinking. We cultivate this skill in games like chess. Great chess masters are considered exceptionally intelligent. The romantic movement has fantasized about chess playing 'Turks'. From there  it has been a long way to the year 1997, when Garry Kasparov was defated by Deep Blue. Today every decent chess program running on a notebook can beat the best human players. What took building a specialized computer 20 years ago has become a commodity. A.I.~in this context is the ability to search over possible game continuations, a gigantic combinatorial problem. Deep Blue could analyse 200 million positions per second. Humans are simply over-powered by this computational speed. Interestingly, this did not have a profound impact on the advancement of A.I. The real world is no chess board. \\

\noindent It took computers almost 20 more years to beat humans at the game of Go. This is because Go is not solvable by brute force alone. It has an enormous branching factor and more than $10^{170}$ possible positions, whereas there may just be less than $10^{80}$ atoms in the universe. What contributed to the AlphaGo success is an ability to perform pattern recognition on board positions that predict their game value.\footnote{That is, whether they are a win for black or white.} One can think of this as some form of intutitive visual judgement about game positions. For some Go playing is an art, a display of intellectual beauty, something spiritual. It thus seems that machines are able to grasp and advance that beauty. The newest version, AlphaZero, does not require human interferences or guidance and can advance purely through self-play. \\

\noindent A key quality of AlphaGo was a perceptive skill, which is based on convolutional neural networks (CNNs), a family of learning algorithms (re-)invented recently. The Go points are like pixels in an image. CNNs have led to very significant progress in computer vision, the paradigmatic case being visual object recognition. Classification errors on a standard benchmark \footnote{ImageNet} has dropped from 1 out of 4 in 2011 to 1 out of 33 in 2016. Similar progress is reported on many other perception tasks. Computers now challenge human recognition capabilities and are excelling on specialized tasks. For instance, they have become really good at reading lips and sign language.  Computers can also upgrade their senses/-ors and do things that are totally out of reach for humans, for instance, hear music in the vibrations of a chips bag or the leaves of a plant.\\

\noindent Seeing requires intelligence as the complexity of scenes is enormous and the resulting pixel space is gigantic. Being able to perceive\footnote{This may include other 'non-human' sensors such as laser range finders, sonar, radar etc.} the environment, however, is the foundation for autonomy in the real world. For A.I.~to make sense of the real world, it has to start with the senses or sensors. Clearly, machine perception is the fundament for self-driving cars as well as self-flying micro-aircrafts or drohnes.  Seeing and sensing often need to be closely combined with control. Our hands and eyes often operate as one and the same is true for autonomous machines. Some physical machines are so complex that it requires highly complex controllers to operate them. A human may not be able to control the 40 or so engines of an Illium plane individually and optimally, but an intelligent controller can. We are very slow at learning new motor skills (e.g.~skiing, piano), machines are not (necessarily). \\

\section{We, Humans}

\noindent Our face is what we use as much to display our inviduality as to express ourselves. Computers have become way better than humans in recognizing faces. How can they not be? Facebook receives about 1 billion photos per day\footnote{As many photos per year as there are stars in the milky way.}, many of which contain faces and many of which come with annotations and contextual information. This is a feat for CNNs. It is estimated that a person sees 3 million faces on average, but only memorizes 3000 of them\footnote{www.quora.com/On-average-how-many-faces-does-a-human-being-see-in-his-her-lifetime}. Do you notice a difference? When we say that such an A.I.~recognizes faces, we are actually using an incredibly misleading metaphor. What if Facebook's system would know everyone on this planet from their face (with a bit of contextual information)? It is not hard to imagine what happens, if A.I.~technology pairs-up with existing technology to boost its potentcy. Think of surveillance cameras that have largely been used for real-time monitoring by humans or restrospective analysis in case of an incident. Can you so the difference of what happens as we start connecting face recognition A.I.~to such a network of cameras? Changing the software behind the hardware is a game-changer. Sometimes A.I.~can see things that we may want to hide or at least have control over revealing. The recent study by Michal Kosinski and co-workers shows how the same face recognition systems can also predict the sexual orientation of a person with super-human accuracy and that although evolution has probably highly optimized our face recognition skills. By the way, did I say that we will never get the Genie back into the bottle? There are other, more fun, surprises. One can also go the other way and go from `thoughts' to images in machines. It's like a machine dreaming. It may dream of faces or bed room scenes.\footnote{Insert picture here.} Or what you may ask it to dream of. \\

\noindent For many millenia, we have talked among humans. We also sometimes try to talk to animals. We may have prayed to the Gods or fore-fathers. We may have talked to ourselves. A few years ago, we started talking to machines and future generation will deem that totally normal. Recognizing our voices and understanding what we say seems like a signal processing problem, but it is a formidable excercise in inference and dealing with noise and uncertainty. What we really hear is far less intelligible than we think. We constantly use our knowledge of what constitutes a reasonable (to be expected) sentence or utterance, for instance. Without it, we could not disambiguate the noisy acoustics the way we can. Again, deep learning -- in particular through the use of recurrent neural networks -- has greatly advanced the state-of-the art, cutting word recognition errors by more than a half. And remember, Siri talks to a good fractiion of the world population and it can learn from that! In the not too distant future we will start talking to machines like Amazon's Alexa in our home, only that Alexa will talk to everyone in every home. Similar to vision, we can also invert the process and generate speech from text. Finally, you will like the A.I.~voice that you hear as much as you may like the one in the movie Her.\\


\section{Sense from Language}

Talking about language. Understanding speech is one thing, but understanding language another one. Philosophers have asserted for centuries that there is a close connection in how we -- as humans -- think and how we make use of language. Wittgenstein was most radical by claiming that the meaning of language can only be found in its usage.  Language is hard one though. Yann LeCun calls \textit{Natural Language Processing} the next frontier in A.I. However, note that there is a tradeoff between depth of understanding (of a text, say) and the breadth of machine reading. Google's system can easily process (`read'?) 100s of billions of pages. It can therefore do things, that no human can do. Google's web search is more of an A.I.~then you may think. Again, we may just have different expectations of what A.I.~will look like. Can you imagine that you would be asked 10 billion questions per day by half of the world's population? \\

\noindent There is also significant progress towards deeper language and text understanding. Watson's performance on the Jepoardy! challenge in 2011 is one prominent examples. Sifting through a few 10 million documents in realtime may be enough of a competitive advantage relative to the best human wizards. Recent progress has been fueled by what is known as embeddings, representations of words and  sentences as activiation vectors. inspired by neural models. Billions of facts about the world are compiled into knowledge bases and linked up with Web pages, articles and books to create an unprecedented web of knowledge. This will shape the future of search engines, but also offers new ways to perform inferences.

\end{document}


\noindent 


\noindent 

\noindent Embeddings. Representation of words, sentences, thoughts. They are abstract, not like perception. No natural mode of representation. Metaphor of neural activity. Something we can do math on top.

% \noindent Sentiment analysis. \\

\noindent Machine translation. Do you speak a 100 languages decently? \\ 

\noindent Question answering and dialogue. Can we get systems to produce natural utterances?  This is a stroy that will unroll in our homes. \\

\noindent Referential semantics. \\

\noindent What would you do with a system than could read as many documents as you would want it to read with  some decent level of understanding? \\

\section{Sense from People}

\noindent We have already touched on some of it... Interpreting faces and facial expressions. What is in your face, your eyes, your gaze? What does the way you walk tell about you? What is in your voice? \\

\noindent But also: how much can we know about a person from the use of the internet or apps? Good thing: recommender systems, etc. 1plusX.\\

\noindent But also: how much do you reveal about yourself in social networks. Cambridge analytics. 50 likes arw sufficient to characterize your personality at the level of what a close friend or family member could. \\

\noindent Benefits: personalized health and precision medicine. Food etc. We are all different. Intelligent computers can attend to our differences.\\ 

\noindent AI-Commerce\\

\noindent Health, genetics.\\

\noindent Health, hospital, emergency room, etc. \\

\noindent Insurance, citizen score etc.\\

\section{Sense from Data}

Common ML approach. What is your data? What is your task? Predictive maintenance. Optimal control. Traffic prediction. Automated decison making in administrative processes. Better use of resources like energy. Less food waste. Etc. etc. Pattern is always the same: deploy sensors and collect lots of data, learn a statistical model, etc. 

\section{Closing}

\noindent  We have opened the bottle. There is no way that we get the Genie back into the bottle. \\

\end{document}

Multifaceted, 

We send our children to school 


Intelligence is the ability to understand and to act accordingly. This involves pattern recognition, statistical and causal models,  

\newpage

\section{Closing}

\end{document}

Intelligence <= Processing of information 
Information & Knowledge: hidden in "data"
Intelligence => Super-human Intelligence 

