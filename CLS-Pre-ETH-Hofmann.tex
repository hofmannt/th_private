\documentclass{article}
\author{Thomas Hofmann}
\title{Center for Learning Systems \\ Preface -- ETH}
\begin{document}
\maketitle

The Center for Learning Systems is, by design, an international and inter-institutional endeavor,  but also a widely cross-disciplinary effort. Many of the biggest scientific opportunities and most impactful innovations around machine intelligence and learning systems require expertise from multiple disciplines. At ETH Zurich this has been mirrored by the organizational challenge to bring together, not just researchers from MPG and ETHZ, but to also coordinate researchers within ETHZ that are associated with different departments, most notably, Computer Science, Information Technology and Electrical Engineering, Mechanical and Process Engineering, and Mathematics, plus a sizable number of additional departments. In order to avoid drawing pre-defined, static boundaries, e.g.~about membership in the Center and distribution of funds, we have taken a participation-based approach and prioritized research projects, where it was clear that colleagues from both institutions would engage in meaningful collaborations and for which our carefully selected Ph.D.~students showed most excitement. We have found this bottom-up, partially self-organizing decision-making model, in which we include top-talented Ph.D.~students, who make an important choice for their professional life, as well as the implicit preferences  behind the match-making between members from MPI and ETHZ, to be highly effective in selecting projects. It has helped us to avoid the overheads of proposal writing and of organizing formal review processes. This grassroots approach has also helped tremendously to build a vivid CLS community, something that has been further supported by annual off-sites and a multitude of events such as targeted workshops and summer schools. Building such a community has required time, effort and patience, but today we see the fruits of this in the established network and the productive exchange of Ph.D.~students and researchers. There has been a growing identification of members and fellows with the Center on the inside and an increased interest in CLS positions on the outside, resulting in a candidate pool that stands out, even relative to the selective programs at both institutions. As it has been our core principle that excellent, groundbreaking research requires exceptional talent, we feel that we have met -- beyond the specifics of projects and collaborations -- the main goal we had set for ourselves: to attract the best of the best and to provide them with a stimulative and engaging environment to grow and to become maximally productive.  

\end{document}