\documentclass{article}
\usepackage{amsmath,amsfonts,amsthm}
\author{Thomas Hofmann}
\begin{document}
\begin{itemize}
\item Given $p$, an implicant of $f$, i.e.$p(\alpha) \le f(\alpha)$ for all $\alpha \in \{0,1\}^n$. 
\item Assume $p = q \cdot x_i$ (i.e.~it has a positive literal)
\item We will shop that $q$ is a (smaller) implicant (thus $p$ cannot be prime)
\item Case distinction
\begin{enumerate}
\item $\alpha$ s.t.~$\alpha_i=1$: then $p(\alpha) = q(\alpha)$ and thus $q(\alpha) = p(\alpha) \le f(\alpha)$. 
\item $\alpha$ s.t.~$\alpha_i=0$
\begin{enumerate}
\item if $q(\alpha)=0$, then $q(\alpha) = 0 \le f(\alpha)$ trivially  
\item if $q(\alpha)=1$, then $f(\alpha) \ge f(\alpha|  \alpha_i \leftarrow 1) \ge p(\alpha|  \alpha_i \leftarrow 1) = q(\alpha)$ 
\end{enumerate}
\end{enumerate}
\end{itemize}

\end{document}