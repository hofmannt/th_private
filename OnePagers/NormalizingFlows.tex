\documentclass[twocolumn]{article}

\usepackage{amsmath}
\usepackage{graphicx} 
\usepackage{subfigure} 

\usepackage{algorithm}
\usepackage{algorithmic}
\usepackage{hyperref}

\newcommand{\x}{{\mathbf x}}
\newcommand{\y}{{\mathbf y}}

\title{Normalizing Flows} 
\author{Thomas Hofmann, ETH Z\"urich}

\begin{document} 

\maketitle 

How can one transform a given (simple) probability density into a family of more complex densities? A straightforward approach is to transform the random variable in question via a differentiable and  invertible map $F: \x \mapsto \y$ which induces a new density via the rule of change of variables, i.e.
\begin{align}
p_\y(\y) = p_\x(F^{-1}(\y))  \cdot |\text{det} \, \mathbf J_F(\x)|^{-1}\,.
\end{align}
In cases, where this map belongs to a parameteric family, one would like to avoid having to deal with the Jacobian by constructing families $\mathcal F$ such that \textit{by design} $F\in \mathcal F \Longrightarrow \text{det}  \, \mathbf J_{F}(\x)= \pm 1$. In fact, one can then apply the transformation repeatedly to generate a \textit{normalizing flow} $\y^t = F(\y^{t-1}) = (F^t)(\x)$, $t \ge 1$ \cite{rezende2015variational}. Obviously, if $\mathcal F$  is a $d$-dimensional parameteric family, we will create a $d$-dimensional family of probability distributions. However, as opposed to standard approaches that create homogeneous families, e.g.~by pre-scribing sufficient statistics as in exponential families, the idea is to create a more diverse set of distributions, possibly including multimodal and differently shaped densities.

\newpage


One approach is to repeatedly apply to not solve this problem in one step, but to  generate a sequence of densities by repeated application of the same transformations and create a so-called (normalizing) flow. 

\newpage


One can easliy construct volume preserving transformations by using auto-regressive $F: \x \mapsto \y$ in which $y_i =x_i + f_i(x_{1:i-1})$ such that irrespective of the form of the functions $f_i$ one gets $|\mathbf J_F|=1$, an idea that goes back to \cite{deco1995nonlinear}.



\bibliography{../th.bib}
\bibliographystyle{acm}

\end{document} 
