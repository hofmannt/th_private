\documentclass{article}
\usepackage{amsmath,amsfonts,amsthm}
\begin{document}
\begin{enumerate}
\item SVD of end-to-end linear map 
\begin{align}
W= W^L  \cdots W^1, \; W = U \Sigma V^\top, \; \sigma(W) = \sigma_1 = u_1^\top W v_1
\end{align}
where $u_1,v_1$ are the principal l/r singular vectors.\\
\item For simplicity look at partial derivative w.r.t.~weight $a=W^l_{ij}$. Denote $\bar v := W_{l-1} \cdots W_1 v_1$ and $\bar u :=  u_1^\top W_L \cdots W_{l+1}$. We get by the product rule
\begin{align}
\partial_a \sigma(W) = 
	\bar u^\top \partial_a W \bar v
	+ 
	\sigma  \langle \partial_a u,  v\rangle
	+
	\sigma \langle u, \partial_a v \rangle
\end{align}
\item The first term has the most direct influence on the gain 
\begin{align}
\partial_a W = \mathbf I_{ij}, \quad 
 \bar u^\top \partial_a W \bar v= \bar u_i \bar v_j
\end{align}
with $\mathbf I_{ij} $ has a one at $(i,j)$ and is zero elsewhere. Side note: once singular vectors are known, $\sigma_1$ is the product of gains across layers. 
\begin{align}
\sigma^l = \frac{| \bar u^l W^l \bar v^l |}{\|\bar v^l\|}
\end{align}
\item The other contributions measure effects due to changes (i.e.~rotations) of the singular vectors. We can compute as follows (non-formal notation):
\begin{align}
v+dv & = W_{1}\dots W_{l-1}  \left[ W_{l} + da \right] W_{L} u, \quad \text{s.t.} \; d\|v\|^2 =0 \\
u+ du & = ...
\end{align}
\end{enumerate}

\end{document}