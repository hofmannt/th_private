\documentclass{article}
\usepackage{hyperref}
\author{Thomas Hofmann}
\title{Center for Learning Systems \\ New Structural Measures}
\begin{document}
\maketitle 

\section{Future Directions}

\label{sec:future_directions}
% \richard{Very mixed content here, much of it needs developing into full text.}
\subsection{Recent Developments}

Over the last years, the area of learning systems has seen a very significant increase in public interest. In the public discourse, the term \textit{Artificial Intelligence} (AI) is often use to refer to the design of intelligent systems, which without any noteable exception make use of machine learning, most often based on Deep Neural Networks (DNN), in which case one also uses the term \textit{Deep Learning}. So one can largely think of ``AI systems" and ``learning systems" as synonymous. With the CLS, we have managed to anticipate the importance of this area more than 5-6 years back, yet admittedly underestimated the current ``explosion" of interest. Today, big internet companies invest billions into new AI technologies and services \& products based on such technology. Old-economy companies across many industries (e.g.~automotive, robotics, pharma, finance, medical) find themselves under pressure to innovate in ways that often critically depend on AI. There is also global competition around talents, which are scarce in this area, and which has let to a shortage of qualified researchers (junior and senior) in the academic world, due to a lack of comepetitiveness in terms of compensation as well as work environment (e.g.~data, large teams). 

\subsection{New Research Topics}

With the CLS we find ourselves in an excellent position to further grow and intensify the cooperation between Max Planck and ETH, but also to help shape the future of AI in Europe. In terms of research topics, we thus suggest certain adjustments that take these new trends into account, specifically:  
\begin{itemize}
\item We want to continue to invest in foundational research in the areas of machine learning, statistics, computer vision, control theory, small-/large-scale robotics, and the areas and topics pursued in the first phase of the proposal. This line of research is more relevant than ever. 
\item It is important to reflect the success of Deep Learning and to put more emphasis on DNNs, both in terms of advancing our theoretical understanding as well as in terms of their use in applications. DNNs have led to breakthroughs in a wide range of areas from visual object recognition to speech recognition and (better) language understanding. Major breakthroughs in other areas are imminent.
\item We would like to intensify our attempts to translate foundational research results into innovations. This is something that will be reflected in the choice of research projects, but also in better structural measures to support innovation and entrepreneurship.  
\item In the area of robotics, there has been exciting progress on soft-robotics as well as bio-inspired robotics. We would like to put more emphasis on these themes, given that it is well covered by current CLS members. 
\item Reinforcement learning has been a classical area within machine learning, but one, that had limited impact in real-world applications. This has been changing as of recently, where reinforcement learning has become relevant for complex control problems as well as other areas of automated decision making. 
\item  Embedded and device computing with DNNs has become an interesting topic of research, bordering on areas such as hardware design, computer systems, and programming languages. It is also closely related to model compression and approximate computing. We will aim for adding members with expertise in this area. 
\end{itemize}

\subsection{New Structural Measures}

CLS combines Europe's leading computer science department with Europe's leading basic research institute studying intelligent systems. We have created a joint PhD program which is internationally competitive, providing students with the chance to work with some of the strongest AI faculty worldwide. In the next phase of CLS, we want to jointly build a lighthouse for machine learning (ML) and modern artificial intelligence (AI) in Europe.

\begin{itemize} 
\item \textbf{Joint Ph.D.~Program.}
The major feature of the first phase of CLS has been a joint Ph.D.~program, where each Ph.D.~student has one supervisor from both, ETH and MPI. The Ph.D.~positions have been funded at the 50\% level  to best leverage funds and to make sure there is full commitment from the side of the supervisors. We would like to continue with this funding model and continue to support joint Ph.D.~students in the next phase of CLS. We also believe the 3+1 year model of requiring Ph.D.~students to spend at least one year at the respective partner institution has proven to be effective, as it requires substantial commitment from the student's side. 
%
\item \textbf{Joint Affiliations.} Throughout the first phase of the funding period, instruments and legal frameworks have been developed to allow members of the Center to be co-affiliated with both institutions. After changes to the ETH bylaws, MPI directors can now become ETH professors (first case: Bernhard Sch\"olkopf). ETH professors on the other hand can become Max Planck Fellows. In both cases, there affiliations come with funding for additional Ph.D.~students. This increases the bi-lateral involvements of members with the respective partner institution. 
%
\item \textbf{Group Leaders and Postdocs.}  We propose to complement the joint Ph.D.~fellowships by positions at the group leader and postdoctoral levels. In particular, we want to establish two group leader positions, one based in T\"ubingen/Stuttgart and one at ETH. These positions will come with two earmarked positions for CLS PhD students as well as a small budget for travel and expenses. At ETH, the status of these group leaders will be that of non-tenure track Assistant Professors (similar to SNF professorships). The group leaders will help coordinate and lead the Center. \\[2mm] In addition, we envision the creation of postdoctoral fellowships. To not over-extend the base funding for the Center, we propose to find industry sponsors for Postdocs associated with the CLS.
%
\item \textbf{Strategic Meetings.} We suggest to establish high-level strategic meetings on the future of ML and AI (and how this affects the priorities of the Center). These could happen bi-annually: (1) Once per year co-located with DALI, the major European ML venue, co-established with the help of CLS and its members. The focus would largely be to put the CLS in the context of a larger European strategy for AI and ML. (2) Once per year co-located with a major international conference, which would allow inviting key global leaders in the field. 
%
\item \textbf{CLS Senior Fellows.} Though both institutions have world experts in machine learning and intelligent systems, it is beneficial to reach out beyond in order to create a network of external (senior) fellows, mainly tapping into a select group of outstanding European researchers in ML and AI. Following the successful CIFAR program\footnote{\url{https://www.cifar.ca/research/fellows-advisors}}, CLS will elect senior fellows who will regularly participate in the high level meetings of the Center, provide scientific advice and help set the strategic direction. Fellows would receive small annual budgets to support Ph.D.~researchers, which are then expected to interact with CLS researchers. A small number of such fellowships will be sponsored directly from CLS funds, but additional fellowships can be made available by tapping into third party funding sources (e.g.~industrial partners).
%
\item \textbf{Academic Partnerships.} We will seek to build partnerships with centers and programs outside of Europe. In particular, we will partner with the strongest existing cross-site initiative in North America, the program for Learning in Machines and Brains (LMB) funded by the Canadian Institute for Advanced Research (CIFAR). This will strengthen the Center's profile in the recently emerged field of Deep Learning and will provide increased access to what is perhaps the strongest AI talent pool in the world. 
%
\item \textbf{Industry Partnerships,}  In the first phase of the CLS, we have made the conscious choice not to augment funds by industry sponsorships. The primary goal has been to develop a scientific agenda and to build an academic community. In this founding phase, we considered industry partnerships to be inadequate.  Moving forward in the growth phase of the Center, however, there are many opportunities of how to better partner with companies, specifically with those who pursue a strong research agenda in ML and AI themselves. 
%
\item \textbf{Entrepreneur Program.}  A sizable fraction of the research results obtained at the CLS may have a significant potential for innovation and commercialization. We propose to explore following the blueprint of the successful ETH pioneer fellowship program to provide funding for researchers after the completion of their Ph.D.. These 12-18 months fellowships/stipends  (at ETH or MPI) are meant to enable the formation of start-ups. Funding can be provided through the Center's institutions or through third parties (public or private).
%
\item \textbf{Local Networking.} Since the inception of the CLS, a number of other initiatives in the fields of ML and AI have been launched in the T\"ubingen-Stuttgart as well as the Z\"urich area. Most notably, this includes the Cyber Valley Initiative\footnote{https://www.cyber-valley.de/en}, which ties the MPI with local universities and a large group of industry players. In the ETH domain, there is the Swiss Data Science Center\footnote{https://datascience.ch/} as well as a novel initiative on the Foundations of Data Science\footnote{Led by the Mathematics Department at ETHZ.}. While the focus of the Center continues to be the collaboration across our MPI and ETHZ sites, we believe it to be important to also develop ties to these substantial local developments in order to avoid fragmentation and lack of coordination with these developments.
%
\item \textbf{Support for Women.} The selection process of CLS Ph.D.~fellows has aimed for a sufficient representation of women in the candidate pool invited to the selection event. Moving forward we propose to further raise awareness and develop measures to further support women in the field, for instance, by earmarking a certain fraction of positions and fellowships for women. 
%
\end{itemize}

The above vision is ambitious and would be unrealistic without the foundations that we have built during the first funding period. We believe we stand a chance of building the strongest European center for modern AI research. As the economic and societal influence of AI increases, so will likely the investments, and CLS will influence Europe's further development in this field by proving that it is possible to be internationally competitive in spite of massive investments in China and the US. Therefore, we will aim at making the collaborative activity of MPG and ETH in the field of learning systems permanent, beyond the envisioned second phase of funding. 


\subsection{Towards the European Lab for Learning \& Intelligent Systems (ELLIS)}

Background: Computational methods based on Machine Learning and Artificial Intelligence are forming the basis of every-day interdisciplinary research endeavours. However, Europe is currently lacking the high number of excellent research labs that other countries such as the US are offering. Furthermore, the distinction between academic and industrial research is vanishing as the industry is performing basic research while paying higher salaries and allowing more scientific freedom than academia. The industry is dominated by American or Chinese companies. All of these factors contribute to the European ``brain-drain'', leading to more influx of academic staff in American and Chinese academic and industrial programs.

Assets to the Machine Learning and Artificial Intelligence academic landscape in Europe involve free and standardized high school level education. Throughout the years of education there is a high level of support by the EU social systems. Compared to the US, Europe is offering more education within theoretical science not connected to industry. Within Europe, there are abundant possibilities for close collaborations due to easier commutes compared to USA or China.
The lasting scientific independence is ensured by the highly developed individual and public interest research projects. These are supported as diverse research directions, even in the absence of industrial funding. This also leads to closer networking with other sciences for applications of artificial intelligence.
\noindent
Improvements could be made with respect to:
\begin{itemize}
\item More support with administrative tasks.
\item Junior faculty should be able to confer PhD degrees earlier.
\item Junior faculty should have more stability for funding, location and staffing.
\end{itemize}
Finally, academics are always compared globally, however, the EU academic faculty members have a high teaching duty of 50-60 percent. This puts them at a disadvantage with regards to time spent on research and publications.

\noindent
Developments should involve:
\begin{enumerate}
\item Scientific independence when networking or performing research.
\item Common hiring procedure for excellent students of all levels (Master students and doctoral/postdoctoral fellows) at a competitive salary.
\item Possibility of recruiting research engineers that may collaborate closely with the researchers.
\item Fast-track programs to becoming full professor.
\end{enumerate}

According to ELLIS, these developments should be pursued while taking the researchers? need for stability and the fields? rapid development into account.

\newpage


\end{document}

% Thomas: I have left this out. I feel we are proposing too much and it will become difficult to fund all of these and to manage them as well.
\item 4) CLS global fellow/scholar: 
These could be people who already have a postdoc position funded otherwise, and get an additional flexible budget that they can use for visits, plus the right (and expectation) to participate in the events of the network of fellows. (this is inspired by the Cifar global scholars, see \url{https://www.cifar.ca/research/global-scholars} and \url{https://www.cifar.ca/research/global-academy}). 
Postdocs can advise MSc students (and of course co-advise PhD students day-to-day).
Maybe we could add other benefits, e.g., we could promise that we will nominate them for the ?young academy? (but such things take time..)

% Thomas: postdocs are now mentiuoned elsewhere 
\item 10) Explore industry support
We will explore options to get additional funding for students and postdocs.

% Thomas: Maybe too ambitious, I left this one out. 
\item 11) Explore joint hiring
We will explore options for joint tenure(-track) hiring processes on the group leader or higher levels.


