\documentclass{article}
\usepackage{amsmath,amsfonts}
\usepackage{tikz-cd}
\usepackage{amsthm}
\usetikzlibrary{matrix,arrows,decorations.pathmorphing}

\title{{\large ... Thoughts on ...} \\ Variational Autoencoders}
\author{Thomas Hofmann}
\newcommand{\J}{{\mathbf J}}
\newcommand{\Edata}{\E_{q}} 
\newcommand{\Emodel}{\E_{p}} 
\renewcommand{\Re}{{\mathbb R}}
\newcommand{\logl}{{\mathcal L}}
\newcommand{\ZZ}{{\mathbb Z}}
\newcommand{\NN}{{\mathbb N}}
\newcommand{\mZ}{{\mathbf Z}}
\newcommand{\E}{{\mathbf E}}
\newcommand{\z}{{\mathbf z}}
\newcommand{\x}{{\mathbf x}}
\newcommand{\mX}{{\mathbf X}}
\newcommand{\mY}{{\mathbf Y}}
\newcommand{\y}{{\mathbf y}}
\newcommand{\w}{{\mathbf w}}
\renewcommand{\v}{{\mathbf v}}
\newcommand{\smapsto}{{\stackrel \sim \mapsto \,}}
\newcommand{\word}{{\omega}}
\newcommand{\words}{{\Omega}}
\newcommand{\context}{{\sigma}}
\newcommand{\contexts}{{\Sigma}}
\newcommand{\n}{{n}}
\newcommand{\mN}{{\mathbf N}}
\newcommand{\mat}[1]{{\mathbf #1}}
\newcommand{\lrangle}[1]{{\left\langle \, #1\, \right\rangle}}
\newtheorem{theorem}{Theorem}
\newtheorem{proposition}{Proposition}


\setlength\parindent{0pt}

\begin{document}

\maketitle

\begin{enumerate}
\item \textit{Integration by Parts }: 
\begin{align}
\int u dv= u v - \int v du
\end{align}
which can be directly obtained by integrating the product rule of differentiation.
\item \textit{Gaussian Integration}: denote by $p$ the density of a Gaussian with mean $\mu$ and covariance matrix $\mathbf \Sigma$, then 
\begin{align}
\nabla_\mu \E f  & = \int \nabla_\mu p(x)  f(x) dx = - \int \nabla_x p(x) f(x) dx \\
& \stackrel{*}{=} \int  p(x) \nabla_x f(x) dx = \E \nabla_x f
\end{align}
where it remains to check that (with the marginal density $p_i$)
\begin{align}
\int p(x) f(x) \prod_{j \neq i} dx_j = p_i(x_i) \cdot \E_{X|x_i} f
\end{align}
which is $0$ for $x_i= \pm \infty$, irrespective of what the conditional expectation of $f$ works out to be (assuming it is finite). So it seems the main property used is the location shift property.  
\end{enumerate}

\bibliography{th}
\bibliographystyle{acm}
\end{document}