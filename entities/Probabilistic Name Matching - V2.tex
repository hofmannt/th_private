\documentclass{article}
\usepackage{amsmath}
\usepackage{amssymb}
\usepackage{multicol}
\usepackage{array}
\usepackage{color}

%\usepackage{mdframed}
%\usepackage{mathtools}

\title{Probabilistic Name Matching}
\author{Thomas Hofmann}

\begin{document}
\maketitle

\subsection*{Informal Problem Statement}
Given a document with explicit (partial) entity-links. Increase the linking coverage by identifying all named mentions of the linked entities in the document. 

\subsection*{Formal Problem Statement}

We model a document $d$ as a sequence of tokens (no document stucture): $d = w_1,w_2,\dots,w_l$, where each $w_i$ represents a word. We are given non-overlapping token spans $t_1,\dots,t_m$ each defining a contiguous sub-sequence  of tokens.  We denote $t_j := [a_j:b_j]$, where $a_j$ ($1 \leq a_j \leq l$) is the index of the first and $b_j$ ($a_j \leq b_j \leq l$) the index of the last token in the span. 

With each token span, we are given an entity as specified by a link to wikipedia (or other canonical entity pages), i.e.~we have pairs $\{(t_i,e_j): 1 \leq j \leq m \}$, where entities $e_i$ are identified in some id system such as freebase mids. Denote by $E: = \{ e_j: 1 \leq j \leq m\}$ the set of referenced entities. We would like to identify all additional (unlinked) mentions of the entities in $E$, i.e.~additional token spans that reference elements in $E$. 

\subsection*{Generating Candidate Token Spans} 

As a first step we heuristically identify candidate token spans. Since we are only interested in mentions of a small set of entities, namely those in $E$, we only look for candidate matches of names $n$ that can possibly refer to at least one entity $e \in E$
. To that extent we assume that we have a  probabilistic entity-to-name map, i.e.~probabilities $p(n|e)$ {\color{blue} (Shouldn't it be $p(e|n)$ here ? I guess we are interested in the set of all candidate entities for a name, conditioning on the fact that the name already appears in the text.)}, possibly pruned at some threshold $\theta$ for the sake of sparseness. Formally, we look for all token spans $n$ (not intersecting with the known token spans) such that $p(n|e) \geq \theta$  with $e \in E$ and then add a candidate $(n,e)$. Note that we may have a situation, where the same token span can be used as a name for multiple entities in $E$, in which case we will generate multiple candidates. {\color{blue} (Future question: How do we decide if a name $n$ refers to an entity or not? For example, a person's name like "Tom Depp" might not be related to any wikipedia page, but we might errorneus build a candidate ("Depp", wikipedia/Johhny\_Depp))}

Each name-entity pair $(n,e)$ generated above could be a valid reference, but due to the ambiguity of names it could also be that $n$ indeed represents an entity mention, yet that the referenced entity is not $e$. We thus make use of a probabilistic name-to-entity map to understand the ambiguity in the interpretation of the name represented by $n$. Specifically we look at the following likelihood ratio of the probabilities associated with a random mention $M$\footnote{Here we use a programming style notation to refer to random variables associated with a compound event such as a mention.}
\begin{align}
l(n,e) & := \frac{p(M.ent = e | M.name = n)}{\sum_{e' \neq e} p(M.ent = e' | M.name = n)}
\\ & = \frac{p(M.ent = e | M.name = n)}{1-p(M.ent = e | M.name = n)} \nonumber
\end{align}

{\color{blue} (I think the 1 from the above denominator is in fact \\ $p(M$ represents an entity $|M.name = n) = p(\exists M.ent | M.name=n) =$ \\ $ \sum\limits_{e'} p(M.ent =e' | M.name=n)$ which I believe is not always 1 - it can be computed from the Google index data we plan to use. But if this last probability is $<0.5$, shouldn't we suppose that name n is not corresponding to any entity ?)}

\subsection*{Filtering Candidate Token Spans: Formal Derivation} 

The question is: what should be an appropriate rejection threshold for eliminating the candidate $(n,e)$ due to lack of confidence (or high degrees of ambiguity)?  We could require $l(n,e) > 1$ in order to make sure that the probability that the name $n$ refers to $e$ -- in a context-independent manner -- is greater than the probability that it does not refer to $e$ (i.e.~is greater than $1/2$). \\
{\color{blue} (I would say that, in a context-independent manner, the above approach is a bit too restrictive. I am thinking of the following algorithm:\\
- Compute $p(M$ represents an entity $|M.name = n) = p(\exists M.ent | M.name=n) = \sum\limits_{e'} p(M.ent =e' | M.name=n)$ as described above. If this value is $< 0.5$, then we should probably discard this name as not representing any entity mention.\\
- If the above value is $> 0.5$, then let $C_n$ the set of all entities $e$ that appear in candidate pairs (n,e). These entities from $C_n$ should be sorted by $l(n,e)$ and the maximum scored one should be chosen as the referenced entity.\\
)} 

However, the marginal probabilities in the name-to-entity map do not take into account that $e$ is known to be mentioned in the context of the current page, which is the case by assumption since $e \in E$. So we would need to compute probabilities $p(M.ent =e | M.name=t, M.doc.E)$, where $M.doc.E$ is a random variable referring to the set of entities explicitly linked from the document in which the random mention $M$ occurs. Bascially, we would expect the link probability to increase {\color{blue} (comparing with context-independent probabilties)} for entities $e \in M.doc.E$ and to decrease for those $e' \not\in M.doc.E$. Formally, we can model this through Bayes' rule, where the name-to-entity map models the unconditional, prior probabilities and the random variables $M.doc.E$ and the implied events $e \in M.doc.E$ or $e \not\in M.doc.E$ consitute additional evidence to condition on. Making a simplifying conditional independence assumption\footnote{Namely that  $ p(M.doc.E | M.ent=e, M.name=n) =  p(M.doc.E | M.ent=e)$} we arrive at 
\begin{align}
& p(M.ent =e | M.name=t, M.doc.E)  \\
& \propto  p(M.ent =e | M.name=t) p(M.doc.E | M.ent=e) \,. \nonumber
\end{align}

{\color{blue} (The exact formula includes also $p(M.doc.E | M.name)$ which I suppose is considered as a constant)}

Now, if we ignore cross-dependencies between explicit links for different entities (which is incorrect, but yields significant simplifications), we get the product form over binary indicator events
\begin{align}
p(M.doc.E | M.ent=e) & = \prod_{e' \in M.doc.E} p(e' \in M.doc.E | M.ent=e) 
\\ & \cdot \prod_{e' \not \in M.doc.E} p(e' \not\in M.doc.E | M.ent = e) \,. \nonumber
\end{align}
This looks complicated, but in order to apply Bayes rule as above, we only need to factor out the part that really depends on $e$ and absorbe the rest into a multiplicative constant. Using a similar argument above for the independence of events for different entities\footnote{More precisely here: the occurrence of one entity $e$ in the document does not change the probability that some other entities $e'$ is explicitly linked off the page.} we can use the following simplifications
\begin{align}
p(e' \in M.doc.E | M.ent=e) & = p(e' \in M.doc.E), \quad \text{and}\\
p(e' \not\in M.doc.E | M.ent=e) & = p(e' \not\in M.doc.E), \quad \text{for $e \neq e'$} \nonumber
\end{align}
and define a multiplicative constant as 
\begin{align}
C(E) := \prod_{e' \in E} p(e' \in E) \cdot \prod_{e' \not \in E} p(e' \not\in E)
\end{align}
so that 
\begin{align}
p(M.doc.E | M.ent=e) & = C(M.doc.E) \\
\times & \begin{cases}
\frac{p(e \in M.doc.E | M.ent=e)}{p(e \in M.doc.E)} & \text{if $e \in M.doc.E$} \\
\frac{p(e \not\in M.doc.E | M.ent=e)}{p(e \not\in M.doc.E)} & \text{if $e \not\in M.doc.E$} 
\end{cases} \nonumber
\end{align}
%This formal derivation yields a very intuitive result: For entities that are explicitly linked off a page, the probability of being referenced again will increase by a factor that is a ratio of the conditional and marginal probabilities, assuming that this ratio in general is indeed $>1$. For entities not explicitly linked off the page, the ratio will in general be $<1$.

\subsection*{Filtering Candidate Token Spans: Practicalities} 

What does that mean more practically? Let us focus on the first case, the correction for entities that are actually linked-off the page (as we expect the second effect to be less significant). The marginal probabilities in the denominator are relatively easy to determine by counting
\begin{align}
p(e \in M.doc.E) \approx \frac{|\{d: e \in d.E\}|}{|\{d\}|}
\end{align}
which is simply the document frequency (in terms of explicit linking) of each entity. It can be approximated based on the corpus. A complementary formula can be used for $p(e \not\in M.doc.E)$. 

The probability in the numerator is somewhat harder to estimate. It is the probability that if an entity was mentioned in a document somwhere (random mention $M$), it would be linked off of that document somewhere (else, typically earlier). If publishers would follow the rule that each occurring entity would be linked-off at least once, then this probability would simply be $1$. However, publishers may not follow that rule strictly, so $p(e \in M.doc.E | M.ent=e)$ may be less than $1$. Some entities, say abstract concepts or less important entities, may not be linked to at all, although they appear on a page. Then this probability can be signfificantly less than $1$, even approaching $0$ for entities that are not linked by anyone. Since we are not trying to resolve references to entities not linked off a document, we have no good way of estimating these quantities. We will simply use a heuristic value, say a constant $0 < R \leq 1$ and optimize it based on some extrinsic evaluation of the overall system quality. For entities $e \not\in E$ the probability in the numerator is the probability that an entity is mentioned on the page, yet not linked-off, so that probability is simply $1-R$, if we make the same simplifications as above and assume that the probability does not depend on the specific entity.   
 
Let us plug everything back together, looking at the special case (for better illustration) of where there is exactly one entity $e \in E = M.doc.E$ associated with a mention name $M.name=n$ and all other possible references are entities $e' \not\in M.doc.E$. Then the posterior probability for this unique $e$ becomes
\begin{align}
p(M.ent=e | M.name=n, e\in M.doc.E=E) =  \frac{\frac R{1-R} \cdot p(n|e)}{\sum_{e' \neq e} p(n|e') \frac{p(e \in E)}{p(e' \not\in E)}}
\label{eq:posterior-approx}
\end{align}
{\color{blue} (Shouldn't the above be $p(e|n)$ and $p(e'|n)$ as from eq (2) ?)}\\

\par {\color{blue} (Also, just to make it more clear for me, is it true that here and in (2) the sum in the denominator includes also $e'$ that are not in E, but have $p(e'|n)$ over a treshold?)}


The formula for cases where there are multiple such entities $e$ is similar. We can use \eqref{eq:posterior-approx} to compute a confidence score (as it represents an approximation of the posterior probability) and threshold it at a desired confidence level. {\color{blue} (Or, as said above, we can use this formula to rank all candidate entities for a name, pick the highest scored entity and, if its score is high enough, choose it as the winning entity; otherwise, simply say that the name does not correspond to any entity (from E).)}

 Note that the multiplicative constant $\frac{R}{1-R}$ can just be folded into the confidence thresholding for practical purposes. {\color{blue} (This means that our system should vary just the treshold and not include R at all.)}


\subsection*{Ambiguties in Token Span Segmentation}

Some ambiguity in interpreting names comes from alternative ways of forming token spans in the first place. In order to address this, we employ the following heuristic. Denote by $n^-$ the single token to the left of a span (name candidate) $n$ and $n^+$ the single token to the right of $n$. We consider all sub-spans of $n^-,n,n^+$ which are not sub-spans of $n$ as additonal candidates $(n,*)$ 
, where $*$ represents the fact that these token spans are not associated with any specific entity. By definition, each of these candidates will either include $n^-$ or $n^+$ or both. 

 Example: "Anne Frank" is linked-off the page and $n=$"Frank" is a mention candidate name identified elsewhere on the page. Let us assume that the subsequent token $n^+=$"Miller". Then we want to consider the possibility that the correct mention is actually $n'=$"Frank Miller". Consdering the name "Frank" as a mention of "Frank Miller" will yield a far too low probability as we have not taken into account the more specific name by subotimal token span selection. 
 
A principled way of dealing with token span segmentation is quite involved as many spans may overlap in complex ways. Here we take a myopic local view in which we simply take all overlapping spans $(n,*)$ as identified above and look up entities $e'$ that have at least a $\theta$-probability to be referred to by the name $n$. Then we simply add these pairs $(n,e')$ to the candidate set as defined above. This effectively means that the sum in the denominator in \eqref{eq:posterior-approx} will be augmented by additional (non-negative) terms, which will reduce the posterior probability for $e \in E$ as we would expect intuitively.  

{\color{blue} (Several questions here:\\
Let's suppose that we assigned the candidate $(n,n^+ ; e')$ for the name $n,n^+$ .\\
- I tend to not agree that the sum in the denominator of 8 will be increased. In the denominator of (8), $p(e'|n)$ might be 0 even if $p(e' | n,n^+)$ is very high (because n alone might not be a name for e').\\
- what happens if $e'$ is not in E or not an entity at all (like here, Frank Miller might not be representing a wikipedia entity, but it is still a name of a different person)? The denominator of (8) should include all entities different than e, regardless of whether they are in E or not, which probably answers the first part of the question, but makes this last approach a bit confusing for me. \\
- Shouldn't we evaluate, in this case, $p(M.ent =e | M.name=n, M.doc.E, n+, n-)$ as a function of $p(M.ent =e | M.name=n)$, $p(M'.ent = e' | M'.name=(n,n^+))$ and $p(\exists M'.ent | M'.name=(n,n^+)) = \sum\limits_{e''} p(M'.ent =e'' | M'.name=(n,n^+))$ ? If you think this might be a good direction, I could try to formalize it more.\\
)}

\newpage
{\color{blue}
\subsection*{Evaluation system}
In order to evaluate our system's annotations, we can first use the following simple approach:\\
- take a training set of webpages P. For each page $p\in P$, a set of mentions $M_p = (name,ent)$ is given. Let $text_p$ be the raw text from the page and let $E_p$ be the set of entities from $M_p$, namely $entities(M_p)$. \\
- for each page $p\in P$, run our automated annotation system on ($text_p$, $E_p$) and compute a set of annotations $A_p$ containing pairs $(name,ent)$. Define as true positives $tp=A_p \cap M_p$ and as false negatives $fn=M_p \setminus A_p$. These would give us the recall by summing over all pages $p\in P$. However, computing the precision (deciding if an annotation from $A_p$ is a false positive) is a bit more tricky, so we'll try to approximate it. \\

\par For approximating the precision of the system we will use the following approach: \\
- let $S_p$ be the set of all token spans $[a,b]$ (as described at the beginning of the paper) that represent names of people or real world entities that are written entirely using upper case letters. More formally, $[a,b]\in S_p$: \\
$\bullet$ all tokens $w_i$ with $i\in [a\ldots b]$ start with an upper case letter. \\
$\bullet$ $w_{a-1}$ and $w_{b+1}$ start with a lower case letter (to guarantee that this token span is in the middle of an English sentence)\\
$\bullet$ there is no other character than space in $text_p$ between any tokens $w_i$ and $w_{i+1}$ with $i\in \{a-1 \ldots b\}$\\
- for each token span $[a,b] \in S_p$ - consider the entity (if it exists) $e\in E_p$ with the highest $p(e | [a,b])$ that is above a fixed treshold $\theta'$(these probabilities will be taken from the Google corpus). Let $X_p$ the set of all these entities paired with their token spans from page p. The assumption we are making here is that $E_p \subset entities(X_p)$ and $X_p$ is the maximum set of mentions from page p that refer to English names of people, events $\ldots$ that are written just with uppercase letters.\\
- run our automated annotation system on ($text_p$, $E_p$) and compute, like before, a set of annotations $A_p$ containing pairs  $(name,ent)$, but discarding from it all names that do not have all their tokens starting with uppercase letters. Define as true positives $tp=A_p \cap X_p$ and as false positives $fp=A_p \setminus X_p$. These would give us an approximation of the precision by summing over all pages $p\in P$. 
}
 
\end{document}