\documentclass[german]{article}

\title{
    Mensch--Maschine Interaktion 
    \\ {\Large Leitfragen und Antworten } 
   \\  {\large Innovationsdialog zwischen Bundesregierung, Wirtschaft und Wissenschaft} 
}
\author{Thomas Hofmann, ETH Z\"urich}

\begin{document}
\maketitle

\begin{enumerate}

\item \textit{Was ist Ihre Vision und Zielvorstellung bzw. Ihr Leitbild f\"ur ein gutes Verh\"altnis von Mensch und Maschine?}\\[3mm]
Assistenz, Antizipation, Verstehen ohne Bevormundung und Manipulation 

\item \textit{Was sind besondere St\"arken des deutschen Wissenschafts- und Wirtschaftsstandortes in Bezug auf MMI? Wo sehen Sie Optimierungspotenziale? Wo steht Deutschland im internationalen Vergleich (z.B. USA, Japan und S\"udkorea)?}\\[3mm]
Gutes allgemeines Bildungs- und Innovationspotential, speziell in der Automobilindustrie und der Robotik. Nachteilig: Fehlen von gro�en Internetfirmen. Daten sind Schl�ssel zur Entwicklung neuer MMIs. Solche Daten Entstehen zum Teil bei der Nutzung von Dienstleistungen wie Suchmaschinen und sozialen Medien. Hier haben US Firmen die Nase vorne. Allgmeiner Nachteil: Zur\"uckhaltung und fehlende, "verr\"uckte" Visionen

\item \textit{Welche Forschungs- und Technologiefelder sind f\"ur die Weiterentwicklung der MMI unmittelbar, welche perspektivisch entscheidend? Welche Fortschritte k\"onnten in 5, 10 oder 20 Jahren erreicht werden?}\\[3mm] Maschinelle Intelligenz, getrieben von maschinellem Lernen und grossen Datenmengen (Machine Learning). Text- und Sprachverstehen, F\"ahigkeit zum Dialog (Natural Language Understanding). Interpretation visueller Szenen, Objekterkennung (Machine Vision). Kontextmodelle und Pragmatik (Computational Psychology), Wissensrepr\"asentation und Inferenz (Knowledge Representation), eingebettete Sensorik (Pervasive Computing), computer-unterst\"utztes Lernen (MOOGs etc.)
\item \textit{Welche spezifischen Eigenheiten und Potenziale der MMI sehen Sie in den Bereichen Mobilit\"at, Gesundheit und Industrie?}\\[3mm]
Mobilit\"at: Interfaces mit situativer Intelligenz. St\"andige Begleiter, Push-Technologien, Empfehlungssysteme  (Weiterentwicklungen des Smartphones), autonome Systeme (z.B.~selbst-fahrende Autos, Drohnen)\\[1mm]
Gesundheit: kontinuierliche Vermessung von K\"orpersignalen und deren Interpretation. Assistenz f\"ur Menschen mit eingeschr\"ankten K\"orperlichen oder geistigen F\"ahigkeiten. Diagnosesysteme zur Unterst\"utzung von medizinischem Personal \\[1mm]
Industrie: neue Welle der Verdr\"angung menschlicher Arbeitskraft durch Automatisierung, Produktivit\"atssteigerungen
\begin{itemize}
\item \textit{Zur L\"osung welcher gesellschaftlichen Herausforderungen k\"onnen MMI-Anwendungen beitragen, wo eignen sie sich weniger?}\\[3mm] Plus: Verbesserter Zugang zu Informationen und Wissen. Kognitive Entlastung des Menschen in Arbeitswelt und Freizeit. Neue Qualit\"at der Vernetzung. \\[1mm]
Minus: MMI wird neue Probleme schaffen! Verlust von Arbeitspl\"atzen,  Verlust von Autonomie, Manipulation und \"Uberwachung 
\item  \textit{Wo gibt es bedarfsorientierte, wo technologiegetriebene Entwicklungen, durch die neue M\"arkte oder neue Gesch\"aftsmodelle in etablierten M\"arkten entstehen k\"onnten?}\\[3mm] Wir kennen den Bedarf des Menschen nicht, vor allem nicht den zuk\"unftiger Generationen. "Gadgets" wie Smartphones oder Tablets und Dientleistungen wie Websuche, Social Sharing, YouTube, E-commerce usw.~sind innerhalb von wenigen Jahren zu einem festen Bestand geworden, der nicht mehr wegyudenken ist. Gerade in Bezug auf MMIs wird die Unterscheidung hinf\"allig. Wir werden eine Co-Evolution von Technik (nach deren Gesetzen) und Nutzern (nach Gesetzen der Psychologie und Soziologie) erleben. The possibilities are infinite!
\item \textit{Wovon h\"angt die Akzeptanz neuartiger MMI-Anwendungen in unterschiedlichen Lebensbereichen z.B. am Arbeitsplatz oder im privaten Bereich ab? Gibt es hier gute Fallbeispiele, an denen sich die Einf\"uhrung neuer Technologien orientieren sollte?}\\[3mm]
Gute MMIs zeichnen sich durch die Intuitivt\"at der Nutzung aus. Jeder kann eine Suchmaschine benutzen. Schon Kleinkinder verstehen die Bewegungs-Metaphorik eines Touchscreen-Ger\"ats. Es gibt aber auch viele Beispiele, wie man es falsch machen kann: Aufdringlichkeit der Technologie (MS Paperclip), fehlende Robustheit (brittleness), soziale Normen (speaking in public), Creepiness 
\end{itemize}
\item \textit{Welche ethischen und gesellschaftlichen Fragen sehen Sie durch Fortschritte in der MMI aufgeworfen und noch nicht beantwortet? Von wem erhoffen Sie sich hier Antworten?}\\[3mm]
1.~Rechtliche Fragen: wer tr\"agt die Verantwortung f\"ur Fehler und Fehlentscheidungen? \\
2.~Pers\"onlichkeitsschutz, wie vermeiden wir Mikro-Manipulation (nudging) durch Werbung und Autorit\"aten, Aussp\"ahen durch Geheimdienste und Kriminelle? \\
3.~Informationssicherheit wird durch totale Vernetzung weiter unm\"oglich gemacht. \\
4.~Neuartige Waffensysteme, "algorithmic killing" 
\item \textit{Welche drei Entwicklungen im Bereich MMI werden unser Leben in einer Generation am st\"arksten ver\"andert haben?}\\[3mm] 1.~Nat\"urlich-sprachliche Schnittstellen f\"ur Zugang zu allen Arten von Informationen,\\2.~Situative Awareness oder Intelligenz erm\"oglicht neue Formen der intelligenten Interaktion, \\3.~Sensorinterpretation kombiniert mit *Internet der Dinge* erm\"glicht Autonome Roboter, die nur noch minimalen Input von Menschen brauchen (z.B.~selbstfahrende Autos)
\item \textit{Welche drei W\"unsche haben Sie an die Bundeskanzlerin, die Forschungsministerin und den Wirtschaftsminister?}\\[3mm]
Kanzlerin: Das Thema digitale Gesellschaft als absolute Top Priorit\"at ernst nehmen. Expertise unter Entscheidungstr\"agern erh\"ohen bzw.~Auswahl von kompetenten und vorausdenkenden Entscheidungstr\"agern verbessern. Aktuell auf nationaler, wie auf EU Ebene sehr entt\"auschend! Auch notwendig um Balance mit Lobby der Googles und Facebook dieser Welt wiederzuerlangen.\\[1mm]
Forschungsministerin: Das deutsche Forschungsf\"ordersystem ist zu oft auf "Pseudoinnovation" aus. Innovation ersch\"opft sich vielfach in der Entwicklung von Demonstratoren und Prototypen, mit denen sich medial (aber nicht in der komerziellen Welt) punkten l\"asst. Stattdessen sollten neue Modelle untersucht werden, wie Innovation in IT n\"aher am praktischen Einsatz, etwa in Kombination mit Risikokapital erfolgen kann. \\[1mm]
Wirtschaftsminister: Nachhaltige Initiative, um den Vorsprung der USA im Bereich IT und Web aufzuholen. Diese Technologien werden die Welt weit \"uber das hinaus ver\"andern, was wir einst oder heute als Web kennen. Auch eine Ver\"anderung der Unternehmenskultur ist von N\"oten.\\[1mm]
\end{enumerate}


\end{document} 