\documentclass{beamer}
\usepackage{amsmath}
\usepackage{amssymb}
\usepackage{multicol}
\usepackage{array}
\usepackage{mdframed}
\usepackage{mathtools}

\mode<presentation>

\title{\LARGE Structured  Prediction}
\author{\Large Thomas Hofmann\\[1mm] {\footnotesize \texttt{thomas.hofmann@inf.ethz.ch}}}
\institute{Department of Computer Science, ETH Z\"urich\\[2mm] 
Google Engineering Center, Z\"urich}
\date{Computer Science Colloquium \\[1mm] University of Basel, November 28, 2013}

\usecolortheme{lily}
\setlength{\leftmargini}{1pt}
\input{macros}

\begin{document}

%%%%%%%%%%%%%%%%%%%%%%%%%%%%%%%%%%%%%%%%%%%%%%%%%%

\begin{frame}
\titlepage
\end{frame}

%%%%%%%%%%%%%%%%%%%%%%%%%%%%%%%%%%%%%%%%%%%%%%%%%%
%%%%%%%%%%%%%%%%%%%%%%%%%%%%%%%%%%%%%%%%%%%%%%%%%%

%%%%%%%%%%%%%%%%%%%%%%%%%%%%%%%%%%%%%%%%%%%%%%%%%%
\begin{frame}
\frametitle{Google Zurich Plug}

\includegraphics[width=\textwidth]{graphics/GoogleZurich.png}

\end{frame}

%%%%%%%%%%%%%%%%%%%%%%%%%%%%%%%%%%%%%%%%%%%%%%%%%%
\section{Motivation \& Overview }
\frame{\sectionpage}

%%%%%%%%%%%%%%%%%%%%%%%%%%%%%%%%%%%%%%%%%%%%%%%%%%
\begin{frame}
\frametitle{Overview}

\begin{enumerate}
\item $\highblue{\mathbf \Rightarrow}$ Motivation \& Overview 
\begin{itemize}
\item Applications of Structured Predicition
\item Structured Prediction (Challenges) at Google
\begin{itemize}
\item Click-Through Prediction 
\item Embedded {\bf mini talk on Entity Linking} 
\end{itemize}
\end{itemize}
\item Large Margin Model
\begin{itemize}
\item Structured prediction SVM  
\item Margins \& loss functions for structured prediction  
\end{itemize}
\item Oracle-based Algorithms 
\begin{itemize}
\item Cutting plane methods 
\item Subgradient-based approaches 
\item Frank-Wolfe algorithm 
\end{itemize}
\item {\it Decomposition-based Algorithms}
\item Conclusion \& Discussion
\end{enumerate}

\end{frame}

%%%%%%%%%%%%%%%%%%%%%%%%%%%%%%%%%%%%%%%%%%%%%%%%%%

%%%%%%%%%%%%%%%%%%%%%%%%%%%%%%%%%%%%%%%%%%%%%%%%%%
\begin{frame}
\frametitle{Structured Prediction}

Generalize supervised machine learning methods to deal with \highblue{structured outputs} and/or with multiple, \highblue{interdependent outputs}.

\begin{minipage}{0.45\textwidth}
Structured objects such as sequences, strings, trees, labeled graphs, lattices, etc.
\end{minipage}
\hspace*{5mm}
\begin{minipage}{0.45\textwidth}
\includegraphics[width=\textwidth]{graphics/Sketch01.png}
\end{minipage}

\begin{minipage}{0.45\textwidth}
Multiple response variables that are interdependent = collective classification
\end{minipage}
\hspace*{5mm}
\begin{minipage}{0.45\textwidth}
\centering
\includegraphics[width=0.7\textwidth]{graphics/Sketch02.png}
\end{minipage}

\end{frame}

%%%%%%%%%%%%%%%%%%%%%%%%%%%%%%%%%%%%%%%%
\begin{frame}
\frametitle{Jiggsaw Metaphor}

Holistic prediction $\neq$ independent prediction of pieces\\[5mm]   
\begin{minipage}{0.41\textwidth}
\centering
\includegraphics[width=\textwidth]{graphics/Jiggsaw.png}
\end{minipage}
\hspace*{1mm}
\begin{minipage}{0.07\textwidth}
\centering
\includegraphics[width=\textwidth]{graphics/Rightarrow.png}
\end{minipage}
\hspace*{1mm}
\begin{minipage}{0.41\textwidth}
\centering
\includegraphics[width=\textwidth]{graphics/MalagaPuzzle.png}
\end{minipage}
\vspace*{5mm}

It is not just about solving one instance of a puzzle, but learning how to solve a whole class of puzzles.
\vspace*{15mm}
\begin{flushright}
\tiny inspired by Ben Taskar's tutorial
\end{flushright}
\end{frame}

%%%%%%%%%%%%%%%%%%%%%%%%%%%%%%%%%%%%%%%%
\begin{frame}
\frametitle{Natural Language Processing}

\begin{itemize} \setlength{\itemsep}{6pt}
\item PoS tagging, named entity detection, language modeling 
\item Syntactic sentence parsing, dependency parsing 
\item Semantic parsing
\end{itemize}
\vspace*{3mm}

\begin{minipage}{0.40\textwidth}
\centering
\includegraphics[width=\textwidth]{graphics/NLP-sketch.png}
\end{minipage}
\begin{minipage}{0.06\textwidth}
\hspace*{1mm}
\end{minipage}
\begin{minipage}{0.50\textwidth}
\tiny 
\vspace*{-5mm}
\begin{itemize}
\item B. Taskar, D. Klein, M. Collins, D. Koller, and C. Manning, Max-Margin Parsing, EMNLP, 2004 %184
\item R. McDonald, K. Crammer, F. Pereira, Online large-margin training of dependency parsers. ACL 2005 % 455
\item H. C. Daume III, Practical Structured Learning Techniques for Natural Language Processing, Ph.D. Thesis, Univ. Southern California, 2006 % 61
\item R. McDonald, K. Hannan, Kerry, T. Neylon, M. Wells, J. Reynar,  Structured models for fine-to-coarse sentiment analysis, ACL 2007  % 164
\item B. Roark, M. Saraclar, M. Collins: Discriminative n-gram language modeling. Computer Speech \& Language 21(2): 373-392, 2007 % 127
% \item P. Blunsom, Structured Classi�cation for Multilingual Natural Language Processing, Ph.D. Thesis, Univ. Melbourne, 2007 % 5
\item Y. Zhang, S. Clark, Syntactic processing using the generalized perceptron and beam search, Computational Linguistics 2011.
\item C. Cherry, G. Foster, Batch tuning strategies for statistical machine translation, NAACL 2012. %33
\item L. S. Zettlemoyer, M. Collins, Learning to Map Setences to Logical Form: Structured classification with probabilistic categorial grammars, arXiv, 2012 % 211
% \item T. Koo, Advances in Discriminative Dependency Parsing, Ph.D. Thesis, MIT, 2010. % 4 
\end{itemize}
\end{minipage}

\end{frame} 

%%%%%%%%%%%%%%%%%%%%%%%%%%%%%%%%%%%%%%%%
\begin{frame}
\frametitle{Information Retrieval }

\begin{itemize} \setlength{\itemsep}{6pt}
\item Learning to rank, e.g.~search engines 
\item Multidocument summarization 
\item Whole page clickthrough prediction 
\item Entity linking and reference resolution
\end{itemize}
\vspace*{10mm}

\begin{minipage}{0.40\textwidth}
\centering
\includegraphics[width=\textwidth]{graphics/Summarization-Coverage.png}
\end{minipage}
\begin{minipage}{0.06\textwidth}
\hspace*{1mm}
\end{minipage}
\begin{minipage}{0.50\textwidth}
\tiny 
\begin{itemize}
\item Y Yue, T. Finely, F. Radlinski, T. Joachims: A support vector method for optimizing average precision. SIGIR 2007
\item L. Li et al: Enhancing diversity, coverage and balance for summarization through structure learning, WWW 2009.
\item T. Berg-Kirkpatrick, Taylor, D. Gillick, D. Klein: Jointly Learning to Extract and Compress, ACL. 2011
\item R. Sipos, P. Shivaswamy, T. Joachims: Large-margin learning of submodular summarization models, ACL 2012
\end{itemize}
\end{minipage}

\end{frame} 


%%%%%%%%%%%%%%%%%%%%%%%%%%%%%%%%%%%%%%%%
\begin{frame}
\frametitle{Computer Vision}

\begin{itemize}  \setlength{\itemsep}{6pt}
\item Image segmentation 
\item Scene understanding 
\item Object localization \& recognition 
\end{itemize}
\vspace*{6mm}

\begin{minipage}{0.40\textwidth}
\centering
\includegraphics[width=\textwidth]{graphics/localization.png}
\end{minipage}
\begin{minipage}{0.06\textwidth}
\hspace*{1mm}
\end{minipage}
\begin{minipage}{0.50\textwidth}
\tiny 
\begin{itemize}
\item T. Caetano \& R. Hartley, ICCV 2009 Tutorial on Structured Prediction in Computer Vision
\item M. P. Kumar, P. Torr, A. Zisserman, Efficient Discriminative Learning of Parts-based Models, ICCV 2009.
\item  C. H. Lampert,  M. B. Blaschko, T. Hofmann: Efficient Subwindow Search: A Branch and Bound Framework for Object Localization, PAMI 2009
\item A. G.  Schwing, T. Hazan, M. Pollefeys, R. Urtasun: Efficient structured prediction for 3D indoor scene understanding, CVPR 2012
\item A. Patron-Perez, M. Marszalek, I. Reid, A. Zisserman: Structured learning of human interactions in TV shows, PAMI 2012
\end{itemize}
\end{minipage}

\end{frame} 

%%%%%%%%%%%%%%%%%%%%%%%%%%%%%%%%%%%%%%%%
\begin{frame}
\frametitle{Computational Biology}

\begin{itemize}  \setlength{\itemsep}{6pt}
\item Protein structure \& function prediction  
\item Gene finding, structure prediction (splicing)
\end{itemize}
\vspace*{6mm}

\begin{minipage}{0.40\textwidth}
\centering
\includegraphics[width=\textwidth]{graphics/biology1.png}\\
\includegraphics[width=\textwidth]{graphics/biology2.png}
\end{minipage}
\begin{minipage}{0.06\textwidth}
\hspace*{1mm}
\end{minipage}
\begin{minipage}{0.50\textwidth}
\tiny 
\begin{itemize}
\item Y. Liu, E. P. Xing, and J. Carbonell, Predicting protein folds with structural repeats using a chain graph model, ICML 2005
\item  G. R\"atsch and S. Sonnenburg, Large Scale Hidden Semi-Markov SVMs, NIPS 2006. 
\item  A. Sokolov and A. Ben-Hur, A Structured-Outputs Method for Prediction of Protein Function, In Proceedings of the 3rd International Workshop on Machine Learning in Systems Biology, 2008. 
\item  K. Astikainen et al., Towards Structured Output Prediction of Enzyme Function, BMC Proceedings 2008, 2 (Suppl. 4):S2 
\item  G. Schweikert et al, mGene: Accurate SVM-based gene finding with an application to nematode genomes, Genome Res. 2009 19: 2133-2143
\item N. G\"ornitz, C. Widmer, G. Zeller, A. Kahles, S. Sonnenburg, G. R\"atsch: Hierarchical Multitask Structured Output Learning for Large-scale Sequence Segmentation, NIPS 2011
\end{itemize}
\end{minipage}

\end{frame} 

%%%%%%%%%%%%%%%%%%%%%%%%%%%%%%%%%%%%%%%%%%%%%%%%%%
%%%%%%%%%%%%%%%%%%%%%%%%%%%%%%%%%%%%%%%%%%%%%%%%%%

\section{Structured Prediction at Google}
\frame{\sectionpage}

%%%%%%%%%%%%%%%%%%%%%%%%%%%%%%%%%%%%%%%%%%%%%%%%%%
\begin{frame}
\frametitle{Click-Trough Prediction}
\includegraphics[width=\textwidth]{graphics/ClickThrough.png}
\begin{itemize}
\item Probability of click on one element (e.g.~advert, result) depends on page context
\item Structured prediction: generate optimal whole page layout
\end{itemize}
\end{frame}

%%%%%%%%%%%%%%%%%%%%%%%%%%%%%%%%%%%%%%%%%%%%%%%%%%

\begin{frame}
\frametitle{Generalized Second Prize Auction}
\includegraphics[width=1.0\textwidth]{graphics/Auction.png}
\begin{itemize}
\item CTR prediction influences both, ranking and pricing of ads
\end{itemize}
\end{frame}

%%%%%%%%%%%%%%%%%%%%%%%%%%%%%%%%%%%%%%%%%%%%%%%%%%

\begin{frame}
\frametitle{Production System: Logistic Regression}
\begin{itemize}
\item initial training: many months of logs data = O(trillion) events
\item continuously updated in an on-line fashion, streaming data paradigm
\item stochastic gradient ascent on likelihood function
\begin{align*}
\theta^{(t+1)} = \theta^{(t)} + \eta \nabla l (x^{(t)})
\end{align*}
\item adapted L1 regularization (follow the regularized leader): enforce zero parameters 
\item lots of details: learning rates, multiple models, compression, feature hashing, subsampling negative examples, �
\item cf.~{\it \small McMaham et al 2013, Ad Click Prediction: a View from the Trenches}
\end{itemize}
\end{frame}


%%%%%%%%%%%%%%%%%%%%%%%%%%%%%%%%%%%%%%%%%%%%%%%%%%

\begin{frame}
\frametitle{Knowledge Graph}
\includegraphics[width=1.05\textwidth]{graphics/KnowledgeGraph.png}
\end{frame}



%%%%%%%%%%%%%%%%%%%%%%%%%%%%%%%%%%%%%%%%%%%%%%%%%%

\begin{frame}
\frametitle{Freebase}
\includegraphics[width=1.05\textwidth]{graphics/Freebase.png}
\end{frame}

%%%%%%%%%%%%%%%%%%%%%%%%%%%%%%%%%%%%%%%%%%%%%%%%%%

\begin{frame}
\frametitle{Knowledge Graph: Example}
\includegraphics[width=1.05\textwidth]{graphics/Relationships.png}
\end{frame}

%%%%%%%%%%%%%%%%%%%%%%%%%%%%%%%%%%%%%%%%%%%%%%%%%%

\begin{frame}
\frametitle{Ambiguity of Names}
\includegraphics[width=1.05\textwidth]{graphics/Names.png}
\end{frame}

%%%%%%%%%%%%%%%%%%%%%%%%%%%%%%%%%%%%%%%%%%%%%%%%%%

\begin{frame}
\frametitle{Entity Linking}
\begin{itemize}
\item Disambiguate every mention (=entity reference)  on the Web. 
\item Link token spans in documents to knowledge graph entities
\end{itemize}
\includegraphics[width=1.05\textwidth]{graphics/Linking.png}
\end{frame}


%%%%%%%%%%%%%%%%%%%%%%%%%%%%%%%%%%%%%%%%%%%%%%%%%%

\begin{frame}
\frametitle{Identify Canoncial Pages}
\begin{itemize}
\item Collect all canonical pages for an entity (part of human curation process) 
\end{itemize}
\includegraphics[width=1.05\textwidth]{graphics/Canonical.png}
\end{frame}

%%%%%%%%%%%%%%%%%%%%%%%%%%%%%%%%%%%%%%%%%%%%%%%%%%

\begin{frame}
\frametitle{Mining for Names}
\begin{itemize}
\item Extracting entity name (candidates) from anchors  found on Web pages pointing to canonical page 
\end{itemize}
\includegraphics[width=1.05\textwidth]{graphics/Anchors.png}
\end{frame}

%%%%%%%%%%%%%%%%%%%%%%%%%%%%%%%%%%%%%%%%%%%%%%%%%%

\begin{frame}
\frametitle{Converting Relations into Statistical Dependencies}
\includegraphics[width=1.05\textwidth]{graphics/Dependencies.png}
\end{frame}

%%%%%%%%%%%%%%%%%%%%%%%%%%%%%%%%%%%%%%%%%%%%%%%%%%

\begin{frame}
\frametitle{Joint Inference / Prediction}
\begin{itemize}
\item Each mention restricted to candidate set via name detectors
\item Resolve all entity mentions jointly, exploiting known dependencies
\item Weakly supervised learning
\end{itemize}
\includegraphics[width=1.0\textwidth]{graphics/Linked.png}
\end{frame}



%%%%%%%%%%%%%%%%%%%%%%%%%%%%%%%%%%%%%%%%%%%%%%%%%%
%%%%%%%%%%%%%%%%%%%%%%%%%%%%%%%%%%%%%%%%%%%%%%%%%%

\section{Large Margin Model}
\frame{\sectionpage}

%%%%%%%%%%%%%%%%%%%%%%%%%%%%%%%%%%%%%%%%%%%%%%%%%%
\begin{frame}
\frametitle{Structured Prediction}
\begin{itemize}  \setlength{\itemsep}{6pt}
\item Input space $\Xcal$, output space $\Ycal$ 
\item $|\Ycal|=m$ can be large due to combinatorics
\begin{itemize} 
\item e.g.~label combinations, recursive structures
\end{itemize}
\item Given training data $(x_i,y_i) \in \Xcal \times \Ycal$, $i=1,\dots,n$
\begin{itemize}
\item drawn i.i.d.~from unknown distribution ${\cal D}$
\end{itemize}
\vspace*{2mm}
\item Goal: find a \highblue{mapping $F$}
\begin{align*}
F: \Xcal \to \Ycal 
\end{align*}
\vspace*{-5mm}
\begin{itemize}
\item with a small prediction error
\begin{align*} 
\text{err}(F) = {\bf E}_{\cal D} \left[ \triangle(Y,F(X)) \right] 
\end{align*}
\item relative to some \highblue{loss function} $\triangle: \Ycal \times \Ycal \to\Re_0^+$, \\
with $\triangle(y,y)=0$ and $\triangle(y,y') >0$ for $y \neq y'$.
\end{itemize}
\end{itemize}
\end{frame}

%%%%%%%%%%%%%%%%%%%%%%%%%%%%%%%%%%%%%%%%%%%%%%%%%%
\begin{frame}
\frametitle{Examples: Loss Functions}
\begin{itemize}  \setlength{\itemsep}{8pt}
\item Multilabel prediction 
\begin{itemize}
\item $\Ycal = \{-1,1\}^k$
\item $\triangle(y,y') = \frac 12 (k - \langle y, y' \rangle)$ (Hamming loss)
\end{itemize}
\item Taxonomy classification 
\begin{itemize}
\item $\Ycal = \{1,\dots,k\}$, $k$ classes arranged in a taxonomy 
\item $\triangle(y,y') = \text{tree distance between $y$ and $y'$}$  
\item cf.~\cite{Cai04,Binder12}
\end{itemize}
\item Syntactic parsing
\begin{itemize}
\item $\Ycal = \{\text{labeled parse trees}\}$
\item $\triangle(y,y') = \text{\# labeled spans on which $y$ and $y'$ do not agree}$
\item cf.~\cite{Taskar04}
\end{itemize}
\item Learning to rank  
\begin{itemize}
\item $\Ycal = \{\text{permutations of set of items}\}$
\item $\triangle(y,y') = \text{mean average precision of ranking $y'$ vs.~optimal $y$}$ 
\item cf.~\cite{Yue07}
\end{itemize}
\end{itemize}
\end{frame}


%%%%%%%%%%%%%%%%%%%%%%%%%%%%%%%%%%%%%%%%%%%%%%%%%%
\begin{frame}
\frametitle{Multiclass Prediction}

\begin{itemize} \setlength{\itemsep}{6pt}
\item Apply standard multiclass approach to $\Ycal$ with $|\Ycal|=m$. 
\item Define $\Ycal$-family of \highblue{discriminant functions} $\highblue{f_y}: \Xcal\to \Re$, $y \in \Ycal$
\item Prediction based on winner-takes-all rule 
\begin{align*} 
F(x) =\argmax_{y \in \Ycal} \, f_y(x)
\end{align*}
\item Typical: \highblue{linear} discriminants with weight vector $w_y \in \Re^d$ and 
\begin{align*}
f_y(x) := \langle \phi(x), w_y \rangle
\end{align*}
\vspace*{-4mm}
	\begin{itemize}
	\item shared input representation via feature map $\phi: \Xcal \to \Re^d$
	\end{itemize}
\item Trained via one-vs-all or as a single 'machine'
\item References: \cite{Rifkin04,Weston99,Crammer02,Lee04}
\end{itemize}
\end{frame}

%%%%%%%%%%%%%%%%%%%%%%%%%%%%%%%%%%%%%%%%%%%%%%%%%%
\begin{frame}
\frametitle{Multiclass Prediction $\prec$ Structured Prediction}

\begin{itemize} \setlength{\itemsep}{6pt}
\item What happens as $m > n$ ? 
	\begin{itemize} 
	\item Not enough training data to even have a single example for every output. 
	\end{itemize}
\item  Taking outputs as atomic entities without any internal structure does not enable \highblue{generalization across outputs}
	\begin{itemize} 
	\item There is no learning, only memorization of outputs.
	\end{itemize}
\item Need to go beyond the standard multiclass setting and enable learning across $\Xcal \times \Ycal$. Two lines of thought:
\item \highblue{Feature-based Prediction}: extract features from inputs \& outputs, define discriminant functions with those features
\item \highblue{Factor-based Prediction}: decompose output space into variables and identify factors $[${\it not covered in depth here}$\,]$
\end{itemize}
\end{frame}

%%%%%%%%%%%%%%%%%%%%%%%%%%%%%%%%%%%%%%%%%%%%%%%%%%
\begin{frame}
\frametitle{Feature-based Prediction}

\begin{itemize} \setlength{\itemsep}{6pt}
\item \highblue{Joint feature maps}
\begin{align*}
\psi: \Xcal \times \Ycal \to \Re^d \,,\quad k_\psi((x,y),(x'y')) := \langle \psi(x,y), \psi(x',y') \rangle
\end{align*}
to extract features from input-output pairs.  
\item
Canonical construction by crossing features extracted separately from inputs and outputs
\begin{align*}
\psi = \phi_{\Xcal} \times \phi_{\Ycal}: \Xcal \times \Ycal \to \Re^{d_x\cdot d_y}\,, \quad \psi(x,y) := \phi_\Xcal(x) \times \phi_\Ycal(y)\,.
\end{align*}
\vspace*{-5mm}
\begin{itemize}
\item Can be more selective about features crossed (subsets).
\item Other constructions (beyond crossing) are possible.
\end{itemize}
\item
When using inner products one gets the compelling factorization 
\begin{align*}
k_\psi((x,y),(x',y')) = k_{\phi^\Xcal}(x,x') \cdot k_{\phi^\Ycal}(y,y') \,.
\end{align*}
\end{itemize}
\end{frame}

%%%%%%%%%%%%%%%%%%%%%%%%%%%%%%%%%%%%%%%%%%%%%%%%%%
\begin{frame}
\frametitle{Example: Label Sequence (HMM) \cite{Altun03}}

\begin{itemize}\setlength{\itemsep}{6pt}
\item 
Hidden Markov Models: $\Xcal = (\Re^d)^l$, $\Ycal = \{1,\dots,k\}^l$, where 
\begin{itemize}
\item $l:$ length of sequence
\item $k:$ cardinality of hidden variable
\item $d:$ dimensionality of observations 
\end{itemize}
\item First feature template: local observations
\begin{align*}
\psi^1_{c} (x,y) = \sum_{t=1}^l  {\bf 1}[y^t=c] \cdot \phi(x^t) 
\end{align*}
\vspace*{-3mm}
\begin{itemize}
\item adding up all observations that are assigned to same class $c \in \{1,\dots k \}$ 
\end{itemize}
\item Second feature template: pairwise nearest neighbor interactions
\begin{align*}
\psi^2_{c,\bar c} (x,y) = \sum_{t=1}^{l-1}  {\bf 1}[y^t=c] \cdot {\bf 1}[y^{t+1}=\bar c]
\end{align*}
\vspace*{-3mm}
\begin{itemize}
\item counting number of times labels $(c,\bar c)$ are neighbors
\end{itemize}
\end{itemize}
 
\end{frame}

%%%%%%%%%%%%%%%%%%%%%%%%%%%%%%%%%%%%%%%%%%%%%%%%%%%
%\begin{frame}
%\frametitle{Example: Optimizing Ranking \cite{Yue07}}
%
%\begin{itemize}\setlength{\itemsep}{6pt}
%\item Kandall's tau:
%\begin{align*} 
%\tau  = \frac{\text{\# concord.~pairs - \# discord.~pairs}}{\text{\# all pairs}}
%= 1 - \frac{2 \cdot \text{\# discordant pairs}}{\text{\# all pairs}}
%\end{align*}
%\item Output ranking encoded via pairwise ordering 
%\begin{align*}
%\Ycal =  \{-1,1\}^{k \times k}, \quad 
%y^\prec_{ij} = \begin{cases} 1 & \text{if $i\prec j$} \\ -1 & \text{otherwise} \end{cases}
%\end{align*}
%\item 
%Combined feature function
%\begin{align*}
%\psi(x,y) = \sum_{i<j} y_{ij} [ \phi(x_i) - \phi(x_j) ] 
%\end{align*}
%Bipartite case $C^+(x) \cup C^-(x) = \{1,\dots,k\}$, relev./non-relev.~items 
%\begin{align*}
%\psi(x,y) = \sum_{i\in C^+(x)} \sum_{j \in C^-(x)} y_{ij} [ \phi(x_i) - \phi(x_j) ] 
%\end{align*}
%\end{itemize}
% 
%\end{frame}
%

%%%%%%%%%%%%%%%%%%%%%%%%%%%%%%%%%%%%%%%%%%%%%%%%%%
\begin{frame}
\frametitle{Example: Learning Alignments \cite{Joachims09CACM}}

\begin{itemize}\setlength{\itemsep}{6pt}
\item Input: two annotated (protein) sequences $x = (s_a, s_b)$. 
\item Output: alignment $y$ between two sequences $x$
\item Joint features:\\[4mm]
\includegraphics[width=0.8\textwidth]{graphics/ProteinAlignment.png}
\item
Types of features: combinations of amino acid, secondary structure, solvent accessibility; sliding window; PSI-BLAST profile scores;
\end{itemize}
 
\end{frame}





%%%%%%%%%%%%%%%%%%%%%%%%%%%%%%%%%%%%%%%%%%%%%%%%%%
\begin{frame}
\frametitle{Multiclass + Output Features = Structured Predction}

\begin{itemize} \setlength{\itemsep}{6pt}
\item Generalize multiclass prediction and define linear discriminants 
\begin{align*}
\text{multiclass}  \rightarrow f_y(x; w) := f(x,y; w) = \langle \psi(x,y), w \rangle \leftarrow \text{structured}
\end{align*}
\item Parameter sharing across outputs (with same features)   
\item Recover feature-less multiclass by defining  ($1$ out of $m$ encoding)
\begin{align*}
\langle \phi_\Ycal(y), \phi_\Ycal(y') \rangle = \delta_{y y'}
\end{align*}
i.e.~feature vectors involving different classes $y,y'$ are orthogonal.
\item Allows to incorporate prior knowledge into multiclass problems
	\begin{itemize}
	\item Hierarchical classification - encode class taxonomy
	\item Entity reference resolution - encode prior entity names and types  
	\end{itemize}
\item Requires single 'machine' formulation as weight vectors are not separated $\rightarrow$ \highblue{How can we generalize SVMs?}
\end{itemize}

\end{frame}

%%%%%%%%%%%%%%%%%%%%%%%%%%%%%%%%%%%%%%%%%%%%%%%%%%
\begin{frame}
\frametitle{Binary Support Vector Machine}
Convex Quadratic Program (primal)  
\begin{align*}
(w^*, \xi^*) = &  \argmin_{w, \xi \geq 0} {\cal H}(w,\xi) := \frac \lambda 2 \langle w, w \rangle  + \frac 1n \| \xi\|_1\\
\nonumber & \; \text{subject to}\quad y_i \langle w, \phi(x_i) \rangle \geq 1 - \xi_i \quad (\forall i)
\end{align*}
\begin{itemize}  \setlength{\itemsep}{6pt}
\item Examples $(x_i,y_i) \in \Xcal \times \{-1,1\}$, $i=1,\dots,n$
\item Feature map $\phi: \Xcal \to \Re^d$
\item Weight vector $w \in \Re^d$
\item Slack variables $\xi_i \geq 0$
\item Regularization parameter $\lambda \in \Re^+$
\end{itemize}
\end{frame}

%%%%%%%%%%%%%%%%%%%%%%%%%%%%%%%%%%%%%%%%%%%%%%%%%%
\begin{frame}
\frametitle{Margin-rescaled Constraints}
For each instance $(x_i,y_i)$ define $m:=|\Ycal|$ constraints via
\begin{align*}
f(x_i,y_i; w) - f(x_i,y; w) \geq \highblue{\triangle(y_i,y)} - \xi_i \quad (\forall y \in \Ycal)
\end{align*}
\begin{itemize}  \setlength{\itemsep}{6pt}
\item Require correct output $y_i$ to be scored higher than all incorrect outputs $y \neq y_i$ by a margin
\item Adjust target margin for incorrect outputs to be $\triangle(y_i,y)$
\item Provides an upper bound on the empirical loss via
\begin{align*}
\xi_i^* & = \max_{y} \{ \triangle(y_i,y) - [f(x_i,y_i; w) - f(x_i,y;w)] \\
& \geq \triangle(y_i,\hat y) - \underbrace{[ f(x_i,y_i; w) - f(x_i, \hat y; w)]}_{\leq 0 \;\text{for $\hat y$}}   
\geq \triangle(y_i,\hat y) 
\end{align*}
where $\hat y_i := \argmax_y f(x_i,y;w)$ is the predicted output
\end{itemize}
\end{frame}

%%%%%%%%%%%%%%%%%%%%%%%%%%%%%%%%%%%%%%%%%%%%%%%%%%
\begin{frame}
\frametitle{Slack-rescaled Constraints}
For each instance $(x_i,y_i)$ define $m:=|\Ycal|-1$ constraints via
\begin{align*}
f(x_i,y_i; w) - f(x_i,y; w) \geq 1- \frac{\xi_i}{\highblue{\triangle(y_i,y)}} \quad (\forall y \in \Ycal - \{y_i\})
\end{align*}
\begin{itemize}  \setlength{\itemsep}{6pt}
\item Require correct output $y_i$ to be scored higher than all incorrect outputs $y \neq y_i$ by a margin
\item Penalize margin violations proportional to $\triangle(y_i,y)$
\item Provides an upper bound on the empirical loss via
\begin{align*}
\xi_i^* & = \max_{y} \{ \triangle(y_i,y) - \triangle(y_i,y)  [f(x_i,y_i; w) - f(x_i,y;w)] \} \\
& \geq  \triangle(y_i, \hat y) - \underbrace{\triangle(y_i,\hat y)}_{\geq 0}  \underbrace{[ f(x_i,y_i; w) - f(x_i, \hat y; w)]}_{\leq 0}  \geq \triangle(y_i,\hat y) 
\end{align*}
where $\hat y_i := \argmax_y f(x_i,y;w)$ is the predicted output
\end{itemize}
\end{frame}

%%%%%%%%%%%%%%%%%%%%%%%%%%%%%%%%%%%%%%%%%%%%%%%%%%
\begin{frame}
\frametitle{Softmargin, Illustration}

\begin{minipage}{0.3\textwidth}
Geometric sketch 
\end{minipage}
\begin{minipage}{0.65\textwidth}
\includegraphics[width=1.7\textwidth]{graphics/margin.png}
\end{minipage}

\end{frame}


%%%%%%%%%%%%%%%%%%%%%%%%%%%%%%%%%%%%%%%%%%%%%%%%%%
\begin{frame}
\frametitle{Structured Prediction SVM}
Convex Quadratic Program (primal)  
\begin{align*}
(w^*, \xi^*) = &  \argmin_{w, \xi \geq 0} {\cal H}(w,\xi) := \frac \lambda 2 \langle w, w \rangle  + \frac 1n \| \xi\|_1\\
& \; \text{subject to} \quad \langle w, \highblue{\delta\psi_i(y)} \rangle \geq \highblue{\triangle(y_i,y)} - \xi_i \quad (\forall i, \highblue{\forall y \in \Ycal- \{y_i\}}) \\
\text{binary:  }& \text{\!\![ subject to} \quad \langle w, \highblue{y_i \phi(x_i)} \rangle \geq \highblue{1} \hspace*{11mm} - \xi_i \quad (\forall i) \hspace*{15mm} ]
\end{align*}
where $\delta\psi_i(y) := \psi(x_i,y_i) -\psi(x_i,y)$.
\vspace*{3mm}

\begin{itemize}  \setlength{\itemsep}{6pt}
\item Examples $(x_i,y_i) \in \Xcal \times \Ycal$, $i=1,\dots,n$
\item Feature map $\psi: \Xcal \times \Ycal \to \Re^d$
\item Weight vector $w \in \Re^d$
\item Slack variables $\xi_i \geq 0$, regularization parameter $\lambda \in \Re^+$
\item Generalizes multiclass SVM \cite{Crammer02}
\end{itemize}
\end{frame}

%%%%%%%%%%%%%%%%%%%%%%%%%%%%%%%%%%%%%%%%%%%%%%%%%%
\begin{frame}
\frametitle{RepresenterTheorem}

\begin{itemize}
\item 
Denote by ${\cal H}$ and RKHS on $\Xcal \times \Ycal$ with kernel $k$. A sample set ${\cal S} =\{(x_i,y_i): i=1,\dots,n\}$ is given. Furthermore let ${\cal C}(f; {\cal S}')$ be a functional that depends on $f$ only through its values on the augmented sample ${\cal S}' := \{ (x_i,y): (x_i,y_i) \in {\cal S} \}$. Let $\Lambda$ be a strictly monotonically increasing function. Then the solution of the optimization problem $\hat f ({\cal S}) := \argmin_{f \in {\cal H}} {\cal C}(f, {\cal S}) + \Lambda(\|f\|_{{\cal H}})$ can be written as 
\begin{align*}
\hat f (\cdot)  = \sum_{i, y} \beta_{iy} k(\cdot, (x_i,y))
\end{align*}
\item Linear case 
\begin{align*}
\hat w = \sum_{i, y} \beta_{iy} \psi(x_i,y)
\end{align*}
\item {\it \small See: T.~Hofmann, B.~Sch\"olkopf, A.J.~Smola, Kernel methods in machine learning, The Annals of Statistics 2008.} 
\end{itemize}

\end{frame}


%%%%%%%%%%%%%%%%%%%%%%%%%%%%%%%%%%%%%%%%%%%%%%%%%%
\begin{frame}
\frametitle{Deriving the Wolfe Dual (1)}

Lagrangian 
\begin{align*}
{\cal L}(\dots)
= & \frac \lambda 2 \| w\|^2 + \frac 1n \| \xi \|_1 
 -\!\! \sum_{i,y \neq y_i} \!\alpha_{iy} \left[ \langle \delta\psi_i(y), w\rangle -\! \triangle(y_i,y) + \xi_i  \right]
- \langle \xi, \hat \xi \rangle
\end{align*}
Gradient components
\begin{align*}
& \nabla_\xi {\cal L} = \frac 1n - \hat \xi - \sum_{y} \alpha_{\bullet y} \stackrel{!}{=} 0
\; \Rightarrow \;  0 \leq \sum_y \alpha_{iy} \leq \frac 1n \quad (\forall i)
\\
& \nabla_w {\cal L} = \lambda w - \sum_{i,y \neq y_i} \alpha_{iy} v_{iy} \delta\psi_{i}(y)  \stackrel{!}{=} 0
\; \Rightarrow  w^*(\alpha) =\frac 1\lambda \sum_{i,y \neq y_i}  \alpha_{iy} \delta\psi_{i}(y)
\end{align*}
Re-writing in matrix notation as $w(\alpha)^* = Q\alpha$ with
\begin{align*}
Q := (Q_{r,iy}) \in \Re^{d \times n(m-1)}, \quad \text{with} \quad Q_{\bullet,iy} := \frac 1\lambda \delta\psi_i(y)
\end{align*}
\end{frame}

%%%%%%%%%%%%%%%%%%%%%%%%%%%%%%%%%%%%%%%%%%%%%%%%%%
\begin{frame}
\frametitle{Deriving the Wolfe Dual (2)}

Plugging-in solution and exploiting known inequalities  
\begin{align*}
& \min_{\alpha \geq 0} h(\alpha) := \frac 12 \|Q \alpha\|^2 - \langle \alpha, \highblue{\triangle} \rangle
\quad \text{s.t.} \;\; n \highblue{\sum_{y}} \alpha_{iy} \leq 1 \; (\forall i) \\
& \hspace*{7mm} \text{binary:   } [ \; \frac 12 \| \tilde Q \alpha \|^2 - \langle \alpha, \; {\bf 1}\rangle
\quad \text{s.t.} \;\; \hspace*{8.7mm} n \alpha_{i} \leq 1\; (\forall i)\;]
\end{align*}
\begin{itemize} \setlength{\itemsep}{6pt}
\item Quantity: $ n\cdot m$ dual variables instead of $n$
\item Quality: structure of dual is very similar
\item Constraints only couple variables in blocks $\{\alpha_{iy}: y \in \Ycal- \{y_i\}\}$ 
\item Natural factorization of $\alpha \in \Re^{n(m-1)}_{\geq 0} = 
\underbrace{\Re^{(m-1)}_{\geq 0} \times \dots \times \Re^{(m-1)}_{\geq 0}}_{n \text{ times}}$
\item $\alpha/n$ is a probability mass function $\alpha_{iy_i} := 1 - n \sum_{y \neq y_i} \alpha_{i y}$
\item What is a support vector? pair $(i,y)$ with active constraint
\end{itemize}
\end{frame}

%%%%%%%%%%%%%%%%%%%%%%%%%%%%%%%%%%%%%%%%%%%%%%%%%%
\begin{frame}
\frametitle{Linear Case: Representer Theorem}

Looking at the solution $w^*$ we see that 
\begin{align*}
w^* & 
= \sum_{i} \sum_{y \neq y_i} \alpha_{iy} \delta\psi_i(y) 
= \sum_{i} \sum_{y \neq y_i} \alpha_{iy} [ \psi(x_i,y_i) - \psi(x_i,y) ] \\ &
= \sum_{i} \underbrace{\left( \sum_{y \neq y_i}  \alpha_{iy} \right)}_{:=\beta_{iy_i}}  \psi(x_i,y_i) 
+ \sum_{i} \sum_{y \neq y_i} \underbrace{(-\alpha_{iy})}_{:=\beta_{iy}} \psi(x_i,y) \\ & 
= \sum_{i,y} \beta_{iy} \psi(x_i,y)
\end{align*}
as it should be according to the representer theorem.
\end{frame}


%%%%%%%%%%%%%%%%%%%%%%%%%%%%%%%%%%%%%%%%%%%%%%%%%%
%%%%%%%%%%%%%%%%%%%%%%%%%%%%%%%%%%%%%%%%%%%%%%%%%%
%%%%%%%%%%%%%%%%%%%%%%%%%%%%%%%%%%%%%%%%%%%%%%%%%%

\section{Oracle-Based Algorithms}
\frame{\sectionpage}

%%%%%%%%%%%%%%%%%%%%%%%%%%%%%%%%%%%%%%%%%%%%%%%%%%
\begin{frame}
\frametitle{The Challenge}

SVMstruct  QP
\begin{align*}
(w^*, \xi^*) = &  \argmin_{w, \xi \geq 0} {\cal H}(w,\xi) := \frac \lambda 2 \langle w, w \rangle  + \frac 1n \| \xi\|_1\\
& \text{with} \quad (w,\xi) \in \bigcap_{iy} \Omega_{iy}, \quad i=1,\dots,n, \; y \in \Ycal -\{y_i\}\\
& \text{where} \quad \Omega_{iy} := \{ (w,\xi): \langle w, \delta\psi_i(y) \rangle \geq \triangle(y_i,y) - \xi_i\} 
\end{align*}
\vspace*{-4mm}
\begin{itemize}  \setlength{\itemsep}{6pt}
\item Structure of QP is not changed, but number of constraints can be vastly increased relative to binary classification
\begin{itemize}
\item e.g.~if $\Ycal$ is vector of binary labels so that $\Ycal = \{-1,1\}^l$ and $m = 2^l$
\end{itemize}
\item Scalable algorithms for this challenge? 10 years of research!
\end{itemize}

\end{frame}

%%%%%%%%%%%%%%%%%%%%%%%%%%%%%%%%%%%%%%%%%%%%%%%%%%
\subsection*{Perceptron Learning}
\frame{\subsectionpage}


%%%%%%%%%%%%%%%%%%%%%%%%%%%%%%%%%%%%%%%%%%%%%%%%%%
\begin{frame}
\frametitle{Structured Prediction Perceptron}

\begin{itemize}  \setlength{\itemsep}{6pt}
\item \highblue{Michael Collins 2002}, Discriminative training methods for hidden markov models: Theory and experiments with perceptron algorithms \cite{Collins02}
\item Perceptron learning avoids the challenge by only focussing on the worst output at a time
\begin{itemize}
\item instead of enforcing constraints over all possible incorrect outputs
\end{itemize}
\item Standard perceptron algorithm with the following modifications
\begin{itemize}
\item Compute prediction
\begin{align*}
\hat y_i := F(x_i) = \argmax_{y} \langle w, \psi(x_i,y) \rangle
\end{align*}
\item Perform update according to 
\begin{align*}
w \leftarrow \begin{cases}
w + \psi(x_i,y_i) - \psi(x_i,\hat y)  = w + \delta\psi_i(\hat y) & \text{if $\hat y \neq y_i$} \\
w & \text{otherwise}
\end{cases}
\end{align*}
\item Novikoff's theorem and mistake bound can be generalized
\end{itemize}
\end{itemize}

\end{frame}

%%%%%%%%%%%%%%%%%%%%%%%%%%%%%%%%%%%%%%%%%%%%%%%%%%
\begin{frame}
\frametitle{Separation Oracles}
\begin{itemize}  \setlength{\itemsep}{6pt}
\item One idea of the perceptron algorithm turns out to be key: identify the output with the \highblue{most violating} margin constraint
\item We call such a sub-routine a \highblue{separation oracle}
\begin{align*}
\text{Perceptron}           \;\;&	\hat y_i \in \argmax_{y} f(x_i,y; w)\\
\text{Margin re-scaling} \;\;&	\hat y_i \in \argmax_{y} \{ \triangle(y_i,y) - f(x_i,y_i; w) + f(x_i,y; w) \} \\
\text{Slack re-scaling}   \;\;&	\hat y_i \in \argmax_{y} \{ \triangle(y_i,y) [1- f(x_i,y_i; w) + f(x_i,y; w)] \}
\end{align*}
\item Dependent on the method, the separation oracle is used to identify 
\begin{itemize}  \setlength{\itemsep}{6pt}
\item violated constraints (successive strengthening)
\item update directions for the primal (subgradient) 
\item variables in the dual (SMO) 
\item update directions for the dual (Frank-Wolfe)
\end{itemize} 
\end{itemize}
\end{frame}


%%%%%%%%%%%%%%%%%%%%%%%%%%%%%%%%%%%%%%%%%%%%%%%%%%%
%\begin{frame}
%\frametitle{Large Margin Algorithms - Taxonomy \& History}
%\includegraphics[width=\textwidth]{graphics/taxonomy.png}
%\end{frame}
%

%%%%%%%%%%%%%%%%%%%%%%%%%%%%%%%%%%%%%%%%%%%%%%%%%%
\subsection*{Successive QP Strengthening}
\frame{\subsectionpage}

%%%%%%%%%%%%%%%%%%%%%%%%%%%%%%%%%%%%%%%%%%%%%%%%%%
\begin{frame}
\frametitle{Successive QP Strengthening}
\begin{itemize}  \setlength{\itemsep}{6pt}
\item Create sequence of QPs that are relaxations of {\it SVMstruct}.  
\item Feasible domain 
$\Omega =  \bigcap_{iy} \Omega_{iy} \cap (\Re^d \times \Re^n_{\geq 0})$
\item Relaxed QP: same objective, yet $\hat\Omega \supset \Omega$
\vspace*{2mm}
\begin{itemize}  \setlength{\itemsep}{4pt}
\item optimal solution $(\hat w, \hat \xi)$ for relaxed QP will have ${\cal H}(\hat w, \hat \xi) \leq {\cal H}(w^*,\xi^*)$, but possibly $(\hat w, \hat \xi) \in \hat \Omega - \Omega$.   
\item goal: fulfill remaining constraints with tolerance $\epsilon$, $(\hat w,\hat\xi+\epsilon) \in \Omega$
\item why? this would give ${\cal H}(\hat w, \hat \xi+\epsilon) = {\cal H}(\hat w, \hat \xi) + \epsilon \geq {\cal H}(w^*, \xi^*)$.
\end{itemize}
\item Construct strict sequence of increasingly stronger relaxations via
\begin{align*}
\Omega(0) = \Re^d \times \Re_{\geq 0}^n, \quad 
\Omega(t+1) := \Omega(t) \cap \Omega_{i\hat y} 
\end{align*}
where $(i,\hat y)$ is a constraint selected at step $t$ fulfilling
\begin{align*}
\argmin_{(w,\xi) \in \Omega(t)} {\cal H}(w,\xi) \not\in \Omega^\epsilon_{i\hat y}, \quad
\Omega^{\epsilon}_{iy} := \{ (w,\xi): (w,\xi+\epsilon) \in \Omega_{iy} \}
\end{align*}
\end{itemize}
\end{frame}

%%%%%%%%%%%%%%%%%%%%%%%%%%%%%%%%%%%%%%%%%%%%%%%%%%
\begin{frame}
\frametitle{Strengthening via Separation Oracle}
\includegraphics[width=0.4\textwidth]{graphics/cut01.png}
\hspace*{6mm} $\to$ \hspace*{6mm} 
\includegraphics[width=0.4\textwidth]{graphics/cut02.png}
\end{frame}

%%%%%%%%%%%%%%%%%%%%%%%%%%%%%%%%%%%%%%%%%%%%%%%%%%
\begin{frame}
\frametitle{Strengthening via Separation Oracle}
\begin{itemize} \setlength{\itemsep}{6pt}
\item Loop through all training examples (in fixed order) 
\item Call separation oracle for $(x_i,y_i)$ 
\item Concretely for margin re-scaling
\begin{align*}
\hat y_i \in \argmax_{y} \{ \triangle(y_i,y) - f(x_i,y_i; w) + f(x_i,y; w) \} 
\end{align*}
will identify (one of) the most violating constraint(s) for given $i$, provided there are such constraints
\item  We can easily check, whether violation is $>\epsilon$. 
\item Termination at step $T$, if no such constraints exist for $i=1,\dots,n$.
\vspace*{5mm}
\item Significance: can ensure $T \leq O(n/\epsilon^2)$ or (with mild conditions) even $T \leq O(n/\epsilon)$. No dependency on $|\Ycal|$!
\end{itemize}
\end{frame}

%%%%%%%%%%%%%%%%%%%%%%%%%%%%%%%%%%%%%%%%%%%%%%%%%%
\begin{frame}
\frametitle{Strengthening via Separation Oracle; Example ($n=1$)}
\begin{itemize} 
\item<1-> 
\begin{minipage}[t]{0.67\textwidth}
Step 0: $ (\hat w, \hat \xi)  = \argmin_{\Omega(0)} {\cal H}(w,\xi) = (0,0)$
\end{minipage}
\item<2-> 
\begin{minipage}[t]{0.67\textwidth}
Step 1: $(\hat w, \hat \xi)  = \argmin_{\Omega(1)} {\cal H}(w,\xi)$, where 
\begin{align*}
& \hat y   \in \argmax_y \triangle(y_1,y),
\\ & \Omega(1)  = \Omega(0) \cap \Omega_{1 \hat y}
\end{align*}
\end{minipage}
\item<3-> 
\begin{minipage}[t]{0.67\textwidth}
Step 2: $(\hat w, \hat \xi)  = \argmin_{\Omega(2)} {\cal H}(w,\xi)$, where 
\begin{align*}
& \hat y   \in \argmax_y \triangle(y_1,y) - \langle \delta\psi_1(y), \hat w\rangle \\
& \Omega(2)  = \Omega(1) \cap \Omega_{1 \hat y}
\end{align*}
provided that $\Omega^\epsilon_{iy} \cap \Omega(1) \neq \emptyset$.
\end{minipage}
\item<4-> 
\begin{minipage}[t]{0.67\textwidth}
Step 3: $(\hat w, \hat \xi)  = \argmin_{\Omega(3)} {\cal H}(w,\xi)$, where ...
\end{minipage}
\end{itemize}
\end{frame}

%%%%%%%%%%%%%%%%%%%%%%%%%%%%%%%%%%%%%%%%%%%%%%%%%%
\subsection*{Cutting Planes}
\frame{\subsectionpage}

%%%%%%%%%%%%%%%%%%%%%%%%%%%%%%%%%%%%%%%%%%%%%%%%%%
\begin{frame}
\frametitle{Improved Cutting Planes: Motivation}
\begin{itemize}\setlength{\itemsep}{6pt}
\item Successive strengthening (as above) is expensive
\begin{itemize} 
\item only one constraint (for one example) gets added in each step
\item requires re-optimization (= solving a QP) after each such step
\item can warm-start, but still...
\end{itemize}
\item How about, we compute all oracles in parallel 
\begin{align*}
\hat y = (\hat y_1, \dots, \hat y_n) \in \Ycal^n
\end{align*}
\item Derive a strengthening from that $\Omega(t+1) = \Omega(t) \, \cap \Omega_{\hat y} $ 
\item Naively could set $\Omega_{\hat y} := \bigcap_i \Omega_{i \hat y_i}$
\begin{itemize}
\item ... but how would that  give us improved termination guarantees?
\item ... how can we avoid blow-up in number of constraints?
\end{itemize}
\item Instead summarize into a single linear constraint with a single shared slack variable $\zeta\geq 0$. Fulfill margin \highblue{on average} 
\begin{align*}
\sum_{i=1}^n \langle \psi_i(\hat y_i), w \rangle \geq \sum_{i=1}^n \triangle(y_i, \hat y_i) - \zeta
\end{align*}
\end{itemize}
\end{frame}

%%%%%%%%%%%%%%%%%%%%%%%%%%%%%%%%%%%%%%%%%%%%%%%%%%
\begin{frame}
\frametitle{Improved Cutting Planes: Algorithm}
\begin{itemize} \setlength{\itemsep}{6pt}
\item \cite{Joachims09} show that the QP containing all such average constraints for all combinations ${\bf y} \in  \Ycal^n$ is solution equivalent to SVMstruct, if one identifies $\zeta = \| \xi_i \|_1$. \\
\framebox[1.0\width][l]{
\begin{minipage}[t]{0.32\textwidth}
\vspace*{-5mm}
\begin{align*}
& \min_{w, \xi} \frac\lambda 2 \langle w, w\rangle + \highblue{\tfrac{1}{n} \|\xi\|_1} \quad \text{s.t.} \\
& \langle \delta\psi_i(y), w\rangle \geq \triangle(y_i,y) - \highblue{\xi_i} \;\\
& (\forall i, y\in \Ycal) \sim n \cdot m
\end{align*}
\end{minipage}
}
\fbox{
\begin{minipage}[t]{0.40\textwidth}
\vspace*{-5mm}
\begin{align*}
& \min_{w, \zeta} \frac\lambda 2 \langle w, w\rangle + \highblue{\zeta} \quad \text{s.t.}  \\
& \highblue{\textstyle\sum_{i}} \langle \delta\psi_i(y), w\rangle \geq \highblue{\textstyle\sum_{i}} \triangle(y_i,y) \!-\!\highblue{\zeta} \\
& (\forall y \in \Ycal^n) \sim m^n
\end{align*}
\end{minipage}
}
\item \cite{Joachims09} also provide $O(1/\epsilon)$-bounds on the number of epochs 
\begin{itemize}
\item overall runtime $O(n/\epsilon)$ (in the linear case), not counting oracle
\end{itemize}
\item Dual QP optimization
\begin{itemize}
\item one variable for each selected (average constraint), highly sparse 
\item complexity of $O(n^2)$; with reduced rank approx.~$O(nr + r^3)$ 
\end{itemize}
\end{itemize}
\end{frame}

%%%%%%%%%%%%%%%%%%%%%%%%%%%%%%%%%%%%%%%%%%%%%%%%%%
\begin{frame}
\frametitle{Improved Cutting Planes: Experiments}
Experiments from \cite{Joachims09}\\[4mm]
\centering\includegraphics[width=0.95\textwidth]{graphics/nslack.png}
\begin{itemize}
\item \# calls to separation oracle 2-3x reduced 
\item CPU time, 5x-1000x dependent on time spent on QP vs.~oracle \\ $\Rightarrow$ much more efficient usage of optimization time 
\item 1000-10000x fewer support vectors, but not when multiplied by $n$
\item Approximation result $O(1/\epsilon)$
\item Book-keeping overhead for storing \#SVs $\cdot n$ descriptors of size $O(\log m)$
\end{itemize}
\end{frame}

%%%%%%%%%%%%%%%%%%%%%%%%%%%%%%%%%%%%%%%%%%%%%%%%%%
\subsection*{Subgradient Methods}
\frame{\subsectionpage}


%%%%%%%%%%%%%%%%%%%%%%%%%%%%%%%%%%%%%%%%%%%%%%%%%%
\begin{frame}
\frametitle{Subgradient Method for SVMstruct}
\begin{itemize}
\item Can we avoid solving many relaxed QPs? \\ How about a gradient descent flavor method? 
\item We can avoid linearizing (i.e.~rolling out) the constraints.\\ Work directly with (unconstrained) piecewise linear objective
\begin{align*}
w^* = \argmin_{w} \frac \lambda 2 \langle w, w \rangle + 
\frac 1n \sum_{i=1}^n \underbrace{\max_y \{ \triangle(y_i,y) - \langle \delta\psi_i(y), w \rangle \}}_{\text{oracle} \; \highblue{\hat y_i} := \argmax ()}
\end{align*}
\item Compute subgradient, e.g.~via 
\begin{align*}
g = \lambda w + \frac 1n \sum_{i=1}^n \delta\psi_i(\highblue{\hat y_i})
\end{align*}
\item Perform batch or stochastic updates on $w$ (learning rate?)
\item Proposed by \cite{Ratliff07}; see also PEGASOS \cite{Shalev11} 
\end{itemize}
\end{frame}


%%%%%%%%%%%%%%%%%%%%%%%%%%%%%%%%%%%%%%%%%%%%%%%%%%
\begin{frame}
\frametitle{Background: Subgradient Methods}
\begin{itemize}
\item Let $f: \Re^D \to \Re$ be a convex, not necessarily differentiable function. A vector $v \in \Re^D$ is called a \highbblue{subgradient} of $f$ at $x_0$, if
\begin{minipage}{0.42\textwidth}
$f(x) \geq f(x_0) + \langle v, x - x_0 \rangle$ for all $x$
\end{minipage}
\begin{minipage}{0.5\textwidth}
\includegraphics[width=1.4\textwidth]{graphics/subgradient.png}
\end{minipage}
\vspace*{-4mm}
\item Differentiable point $x_0$: unique subgradient = gradient $\nabla f(x_0)$.
\end{itemize}
\end{frame}

%%%%%%%%%%%%%%%%%%%%%%%%%%%%%%%%%%%%%%%%%%%%%%%%%%
\subsection*{Frank-Wolfe Algorithm}
\frame{\subsectionpage}

%%%%%%%%%%%%%%%%%%%%%%%%%%%%%%%%%%%%%%%%%%%%%%%%%%
\begin{frame}
\frametitle{Frank-Wolfe Algorithm}
\begin{itemize} \setlength{\itemsep}{8pt}
	\item Frank \& Wolfe, 1956: {\it An algorithm for quadratic programming}
	\item Minimize linearization at current iterate over corners of  domain
	\begin{align*}
		\text{'new iterate'} := (1-\eta) \cdot  \text{'old iterate'} + \; \eta \cdot \text{'optimal corner'}
	\end{align*}\\[2mm]
	\item Features \vspace*{2mm}
	\begin{itemize} \setlength{\itemsep}{4pt}
		\item {\bf linearity}: linear, not quadratic function minimization in every step
		\item {\bf sparseness}: convex combination of selected corners
		\item {\bf projection-free}: iterates stay in convex domain 
		\item {\bf learning rate}: $O(1/t)$ schedule or via line search
		\item {\bf duality gap}: implicitly computes duality gap
	\end{itemize}
	\item Applied to SVMstruct by \cite{Lacoste13}
\end{itemize}
\end{frame}


%%%%%
%%%%%  Frank-Wolfe, quadratic and linearization contour lines
%%%%%
\begin{frame}

\frametitle{ Frank-Wolfe Algorithm: Quadratic vs.~Linearized} 

\begin{tabular}{m{4cm}m{8cm}}
\begin{itemize} \item Quadratic objective (contour line plot) \end{itemize}
& \includegraphics[width=4.5cm]{graphics/quadratic.png} \\
\begin{itemize} \item Linearized objective  \end{itemize}
& \includegraphics[width=4.5cm]{graphics/linearized.png}
\end{tabular}

\end{frame}

%%%%%
%%%%%  Frank-Wolfe, 3D scheme
%%%%%
\begin{frame}

\frametitle{ Frank-Wolfe Algorithm: Schematic 3D View} 

\begin{center}
\includegraphics[width=6cm]{graphics/frank-wolfe.png}

\vspace*{4mm}
[taken from Lacoste-Julien et al.,  2013]
\end{center}

\end{frame}

%%%%%
%%%%%  Frank-Wolfe, applied to dual SVMstruct 
%%%%%
\begin{frame}

\frametitle{ Frank-Wolfe Algorithm: Dual SVM-struct Objective} 

\begin{itemize} \setlength{\itemsep}{8pt}
\item Dual objective
\begin{align*}
h(\alpha) =  \frac 1 2 \| Q \alpha\|^2 - \langle \triangle, \alpha \rangle 
\end{align*}
\item Gradient
\begin{align*}
\nabla_{\alpha} h (\alpha^*) =  (Q' Q) \alpha^* - \triangle
\end{align*}
\item Linearization
\begin{align*}
\bar h(\alpha; \alpha^*) = \underbrace{h(\alpha^*)}_{\text{=const.}}+  \langle \nabla_{\alpha} h (\alpha^*) , \alpha - \alpha^*\rangle  \underbrace{\leq h(\alpha)}_{\text{convexity}}
\end{align*}
\item Minimization problem 
\begin{align*}
e^* := \argmin_{\{e_r: r=1,\dots,m\}} \bar h(e_r; \alpha^*), \quad \text{with $e_r$: $r$-th unit vector}
\end{align*}
\end{itemize}

\end{frame}

%%%%%%%%%%%%%%%%%%%%%%%%%%%%%%%%%%%%%%%%%%%%%%%%%%
\begin{frame}
\frametitle{Frank-Wolfe Algorithm: Deciphered}
\begin{itemize}
\item  What does the minimization problem over  corners mean?
\begin{align*}
\bar h(e_{y'}; \alpha^*) + const. & =  \langle \underbrace{\nabla_{\alpha} h (\alpha^*)}_{=  (Q' Q) \alpha^* - \triangle}  , e_{y'} \rangle \\
& = \langle Qe_{\y'} , \underbrace{Q \alpha^*}_{=w} \rangle + \triangle(y,y') \\
& = \left\langle \textstyle\sum_{i} \delta\psi_i(y'_i), w \right\rangle + \textstyle\sum_i \triangle(y_i, y'_i)
\end{align*}
so that 
\begin{align*}
\hat y_i  = \argmax_{y'} \{ \langle \delta\psi_i(y'), w \rangle + \triangle(y_i, y') \}
\end{align*}
which is just the separation oracle!
\end{itemize}
\end{frame}

%%%%%%%%%%%%%%%%%%%%%%%%%%%%%%%%%%%%%%%%%%%%%%%%%%
\begin{frame}
\frametitle{Algorithms: Frank-Wolfe, Subgradient, Cutting Plane}
\begin{itemize}\setlength{\itemsep}{6pt}
\item  How does Frank-Wolfe relate to the other methods?
\item FW $\leftrightarrow$ Subgradients: 
\begin{itemize}
\item Same update direction of primal solution $w$ 
\item But: Smarter step-size policy derived from dual (see below) 
\item But: Duality gap for meaningful termination condition (see below)
\end{itemize}
\item FW $\leftrightarrow$ improved cutting planes: 
\begin{itemize}
\item Selected dual variables correspond to added constraints  
\item But: incremental update step vs.~optimization of relaxed QP
\item But: \#SV can be larger due to incremental method, no need to re-formulate SVM struct
\end{itemize}
\item Further advantages
\begin{itemize}
\item Simple and clean analysis
\item Per-instance updates (block-coordinate optimization) 
\end{itemize}
\end{itemize}
\end{frame}

%%%%%%%%%%%%%%%%%%%%%%%%%%%%%%%%%%%%%%%%%%%%%%%%%%
\begin{frame}
\frametitle{Frank-Wolfe Algorithm: Primal-Dual Version}
\begin{itemize}\setlength{\itemsep}{6pt}
\item Apply Frank-Wolfe to dual QP, but translate into primal updates
\item Compute primal update direction (subgradient)
\begin{align*}
\bar w := \frac1\lambda \sum_{i=1}^n  \delta\psi_i(\hat y_i), \quad \bar \triangle := \frac 1n \sum_{i=1}^n \triangle(y_i,\hat y_i)
\end{align*}
\item Perform convex combination update 
\begin{align*}
w^{t+1} = (1-\gamma^t) w^t + \gamma^t \bar w, \quad \triangle^{t+1} = (1-\gamma^t) \triangle^t + \gamma \bar \triangle
\end{align*}
here the optimal $\gamma^t$ can be computed analytically (closed-form line search) from $w^t$, $\bar \triangle$ and $\bar w$
\item Convergence rate: $\epsilon$-approximation is found in at most $O(\frac{R^2}{\lambda \highbblue{\epsilon}})$ steps 
\end{itemize}
\end{frame}

%%%%%%%%%%%%%%%%%%%%%%%%%%%%%%%%%%%%%%%%%%%%%%%%%%
\begin{frame}
\frametitle{Block-Coordinate Frank-Wolfe}
\begin{itemize}\setlength{\itemsep}{6pt}
\item Domain of the dual QP factorizes $\alpha \in {\mathcal S}_{m-1}^n$ (product of simplicies)
\begin{align*}
\alpha = (\alpha_i)_{i=1}^n, \; \; \text{s.t.} \quad \alpha_i \geq {\bf 0} \quad \text{and} \quad \langle \alpha_i, {\bf 1} \rangle = 1
\end{align*}
\hspace*{13mm} \includegraphics[width=0.25\textwidth]{graphics/simplex-alpha01.png} 
\hspace*{-3mm} \includegraphics[width=0.25\textwidth]{graphics/simplex-alpha02.png} 
\hspace*{-3mm} \includegraphics[width=0.25\textwidth]{graphics/simplex-alpha03.png} \\[-3mm]
\hspace{6mm} $\alpha = ($ 
\hspace*{7mm} $\alpha_1$ 
\hspace*{19mm} $\alpha_2$ 
\hspace*{18mm} $\alpha_3$ 
\hspace*{8mm} $\dots\; $) 
\vspace*{5mm}
\item Perform Frank-Wolfe update over each block (randomly selected).
\begin{itemize}
\item single-instance mode: alternates single oracle call and update step 
\item back to successive strengthening, but replace: re-optimization with fast updates
\item convergence rate analysis; duality gap as stopping criterion  
\item excellent scalability 
\end{itemize} 
\end{itemize}
\end{frame}

%%%%%%%%%%%%%%%%%%%%%%%%%%%%%%%%%%%%%%%%%%%%%%%%%%
\begin{frame}
\frametitle{Frank-Wolfe Methods: Scalability \cite{Lacoste13}}

\includegraphics[width=0.95\textwidth]{graphics/FWresults.png}
\begin{itemize}
\item Frank-Wolfe very similar to improved cutting plane  method 
\item Block-coordinate version much faster, better than stochastic subgradient descent 
\item Main caveat: primal-dual version needs to store one weight vector per training instance!! 
\end{itemize}
\end{frame}

%%%%%%%%%%%%%%%%%%%%%%%%%%%%%%%%%%%%%%%%%%%%%%%%%%
\section{Conclusion \& Outlook} 
\frame{\sectionpage}

%%%%%%%%%%%%%%%%%%%%%%%%%%%%%%%%%%%%%%%%%%%%%%%%%%
\begin{frame}
\frametitle{Conclusion \& Outlook} 
\begin{itemize}
\item Structured predicition is a very natural generalization of standard classification methods (here: SVM)
\item Highly relevant applications in many sciences and areas of engineering
\item Traditionally, lack of computational scalability has limited the use of structured prediction
\item Many recent advances (subgradient method, Frank-Wolfe) on optimization methods have addressed scalability 
\item Moving forward: use structured predicion for big data problems!
\end{itemize}
\end{frame}



\small
\bibliographystyle {alpha}
\bibliography{thofmann}
\end{document}


%%%%%%%%%%%%%%%%%%%%%%%%%%%%%%%%%%%%%%%%%%%%%%%%%%
%%%%%%%%%%%%%%%%%%%%%%%%%%%%%%%%%%%%%%%%%%%%%%%%%%
%%%%%%%%%%%%%%%%%%%%%%%%%%%%%%%%%%%%%%%%%%%%%%%%%%

\section{Decomposition-Based Algorithms}
\frame{\sectionpage}

%%%%%%%%%%%%%%%%%%%%%%%%%%%%%%%%%%%%%%%%%%%%%%%%%%
\begin{frame}
\frametitle{Factor Graphs}

\begin{itemize}
\item In many cases of practical interest, the compatibility function naturally allows for an additive decomposition over factors or parts
\begin{align*}
f(x,y) = \sum_{c \in {\cal C}} f_c(x_c, y_c)
\end{align*}
which can formally be described as a factor graph. 
\item In the linear case, this can be induced via a feature decomposition 
\begin{align*} 
\psi(x,y) & = \sum_{c \in {\cal C}} \psi_c(x_c, y_c), \quad \text{such that} \\
f(x,y; w) & = \langle w, \psi(x,y) \rangle = \sum_c  \underbrace{\langle w, \psi_c(x_c,y_c) \rangle}_{=: f_c(x_c,y_c)} 
\end{align*}
\item We typically require that the loss decomposes in a compatible manner
\begin{align*}
\triangle(y,y'; x) = \sum_{c \in {\cal C}} \triangle_c(y_c,y_c';x_c)
\end{align*}
\end{itemize}

\end{frame}

%%%%%%%%%%%%%%%%%%%%%%%%%%%%%%%%%%%%%%%%%%%%%%%%%%
\begin{frame}
\frametitle{Representer Theorem for the Factorized Case}

\begin{itemize}
\item 
Conditions as before but factor structure assumed. Denote configurations for factor $c$ as $z\in \Zcal(c)$.
\item
Representation
\begin{align*}
f(x,y) = \sum_{i=1}^n \underbrace{\sum_{c \in {\cal C}} \sum_{z \in \Zcal(c)}}_{\sum_{c} |\Zcal(c)| \ll |\Ycal|}
\mu_{icz} \underbrace{\langle \psi_c(x_{ic} , z), \psi_c(x_{ic}, y_c) \rangle}_{=:k_c((x_{ic},z),(x_{ic}, y_c))}
\end{align*}
\item Note that this offers the possibility to
\begin{enumerate}
\item define kernels on a {\it per factor} level  
\item use a low-dimensional parametrization that does not need to rely on sparseness
\end{enumerate}
\end{itemize}
\end{frame}

%%%%%%%%%%%%%%%%%%%%%%%%%%%%%%%%%%%%%%%%%%%%%%%%%%
\begin{frame}
\frametitle{Decomposing the Dual QP}

\begin{itemize}
\item 
Dual has the following structure (rescaling by $n$ as appropriate to make $\alpha$ probability mass function) 
\begin{align*}
& \min_{\alpha \geq 0} h(\alpha) := \frac 12 \|Q \alpha\|^2 - \langle \alpha, \triangle \rangle
\quad \text{s.t.} \;\; \sum_{y} \alpha_{iy} = 1 \; (\forall i) 
\end{align*}
\item Introduce \highbblue{marginal probabilities}
\begin{align*}
\mu_{icz} := \sum_{i,y} {\bf 1\!}\left[y_c = z \right] \alpha_{iy} , \quad 
\sum_z \mu_{icz} = \sum_{i,y} \alpha_{iy} = 1
\end{align*}
\item 
Decompose loss (similar for $Q^t Q$)
\begin{align*}
\sum_{y} \alpha_{iy} \triangle(y_i,y) &  = 
\sum_{y} \alpha_{iy} \sum_{c} \triangle_c(y_{ic},y_c)
\\& = \sum_{c,z} \underbrace{\left( \sum_y {\bf 1\!}\left[y_c = z \right]  \alpha_{iy}  \right)}_{=  \mu_{icz}}\triangle_c(y_{ic},z)
\end{align*} 
\end{itemize}
\end{frame}

%%%%%%%%%%%%%%%%%%%%%%%%%%%%%%%%%%%%%%%%%%%%%%%%%%
\begin{frame}
\frametitle{Decomposing the Dual QP (continued)}

\begin{itemize}
\item  Define with multi-index $(icz)$:
\begin{align*}
Q_{\bullet, icz} & := \psi_c(x_{ic},y_{ic}) - \psi_c(x_{ic},z), \quad 
\mu_{{\cal C}} & := (\mu_{icz}), \quad \triangle_{\cal C} := (\triangle_{icz})
\end{align*} 
\item  Factorized QP
\begin{align*}
\mu^*_{{\cal C}} & = \argmin_{\mu_{{\cal C}} \geq 0}\left\{  \frac 12 \| Q \mu_{{\cal C}} \|^2 - \langle \mu_{{\cal C}}, \triangle_{\cal C} \rangle \right\} \\
& \text{s.t. $\mu_{{\cal C}}$ is on the \highbblue{marginal polytope}}
\end{align*}
\vspace*{-4mm}
\begin{itemize}
\item $\mu_{\cal C}$ needs to be normalized and locally consistent (non-trivial). 
\item objective broken up into parts - global view enforced via constraints! 
\end{itemize}
\item Example: \highbblue{Singly connected factor graph}. Local consistency:
\begin{align*}
\sum_{r: (r,s) \in {\cal C}} \mu_{irs} = \mu_{is} \quad (\forall i,s)
\end{align*}
\item For general factor graphs: only enforce local consistency = \highbblue{relaxation} in the spirit of approximate belief propagation \cite{Taskar03}
\end{itemize}
\end{frame}

%%%%%%%%%%%%%%%%%%%%%%%%%%%%%%%%%%%%%%%%%%%%%%%%%%%
\begin{frame}
\frametitle{Conditional Exponential Family Models}
\begin{itemize}
\item Structured prediction from a statistical modeling angle
\item $f$ from some RHKS with kernel $k$ over $\Xcal \times \Ycal$
\item \highbblue{Conditional exponential families}
\begin{align*}
p(y | x; f) & = \exp\left[ f(x,y) - g(x,f) \right], \;\; \text{where} \\
g(x,f) & := \int_{\cal Y} \exp \left[ f(x,y) \right] d\nu(y)
\end{align*}
\item Univariate case ($y \in \Re$), generalized linear models
\begin{align*}
p(y|x;w ) =\exp\left[ y\langle w, \phi(x) \rangle - g(x,w) \right]
\end{align*}
\item
Non-parameteric models, e.g.~ANOVA kernels 
\begin{align*}
k((x,y),(x',y') = y y' k(x,x')
\end{align*}
\end{itemize}
\end{frame}

%%%%%%%%%%%%%%%%%%%%%%%%%%%%%%%%%%%%%%%%%%%%%%%%%%%
\begin{frame}
\frametitle{Conditional Random Fields}
\begin{itemize}
\item Conditional log-likelihood criterion \cite{Lafferty01,Lafferty04}  
\begin{align*}
f^* := \argmin_{f \in {\cal H}}  \frac \lambda 2 \underbrace{\| f\|^2_{\cal H}}_{\text{stabilizer}} - \underbrace{\frac 1n \sum_{i=1}^n \log p(y_i|x_i;f)}_{\text{log-loss}}
\end{align*}
\item 
Optimization methods: 
\begin{itemize}
\item improved iterative scaling \cite{Lafferty01}
\item pre-conditioned conjugate gradient descent,  limited memory quasi-Newton \cite{Sha03}
\item finite dimensional case: requires computing expectations of sufficient statistics ${\bf E} \left[ \psi(Y,x) \right]$ for $x=x_i$, $i=1,\dots,n$.
\begin{align*}
\nabla_w [...] \stackrel != 0 \iff  \lambda w^* = 
\underbrace{\frac 1n \sum_{i=1}^n \psi(x_i,y_i)}_{\text{sample statistics}}  - \underbrace{\frac 1n \sum_{i=1}^n\sum_{y \in \Ycal} \psi(x_i,y) p(y|x_i; w^*)}_{\text{expected statistics}}
\end{align*}
\end{itemize}
\end{itemize}
\end{frame}


%%%%%%%%%%%%%%%%%%%%%%%%%%%%%%%%%%%%%%%%%%%%%%%%%%%
\begin{frame}
\frametitle{Dual CRF}
\begin{itemize}
\item Representer theorems apply to log-loss. Log-linear dual: 
\begin{align*}
\alpha^* = & \argmin_{\alpha \geq 0} h(\alpha) := \frac 12 \| Q \alpha \|^2 + \sum_{i=1}^n \sum_{y \in \Ycal} \alpha_{iy} \log \alpha_{iy}\\
& \text{s.t.} \;\; \sum_{y \in \Ycal} \alpha_{iy} = 1 \quad (\forall i)
\end{align*}
\item Compare with SVM struct \\
\begin{minipage}{0.45\textwidth}
\begin{align*}
& \frac 12 \| Q \alpha \|^2 \\
& + \sum_{i=1}^n \highblue{\sum_{y \in \Ycal} \alpha_{iy} \log \alpha_{iy}}
\end{align*}
\end{minipage}
\begin{minipage}{0.1\textwidth}
\end{minipage}
\begin{minipage}{0.45\textwidth}
\begin{align*}
& \frac 12 \| Q \alpha \|^2 \\
& - \sum_{i=1}^n \highblue{\sum_{y \in \Ycal} \alpha_{iy} \triangle(y_i,y)}
\end{align*}
\end{minipage}
\begin{itemize}
\item same data matrix $Q$ constructed from $\delta\psi_i(y)$ (or $Q^t Q$ via kernels)
\item same $n$-factor simplex constraints on $\alpha$ 
\item entropy maximization, instead of linear penalty (based on loss)
\end{itemize}
\end{itemize}
\end{frame}

%%%%%%%%%%%%%%%%%%%%%%%%%%%%%%%%%%%%%%%%%%%%%%%%%%%
\begin{frame}
\frametitle{Exponentiated Gradient Descent}

\begin{itemize}
\item  Exponentiated gradient descent \cite{Collins08} can be applied to solve both duals (hinge loss and logarithmic loss)
\item General update equation
\begin{align*}
\alpha^{(t+1)}_{iy} & \propto \alpha_{iy}^{(t)} \cdot \exp\left[ \nabla h(\alpha) \right] 
\\
& = \alpha_{iy}^{(t)} \cdot \exp\left[ \lambda \langle w^*, \delta\psi_i(y) \rangle- \triangle(y_i,y)\right] 
\end{align*}
\begin{itemize} 
\item Can be motivated by performing gradient descent on the canonical/natural parameters (and re-formulating in mean-value parameterization)
\begin{align*}
\theta^{(t+1)} = \theta^{(t)} + \delta \theta^{(t)} 
\Rightarrow \alpha^{(t+1)} = \exp[ \langle \psi, \theta^{(t+1)} \rangle ]  = \exp[\langle \psi, \delta \theta^{(t)}\rangle] \alpha^{(t)}
\end{align*}
\item on-line version: generalizes SMO for solving dual problem (when no closed form solution exists)
\end{itemize}
\end{itemize}
\end{frame}



%%%%%%%%%%%%%%%%%%%%%%%%%%%%%%%%%%%%%%%%%%%%%%%%%%%
\begin{frame}
\frametitle{Factorized Exponentiated Gradient Descent}

\begin{itemize}
\item  Work with factorized dual QP: e.g. ~\cite{Taskar03}, SMO over marginal variables $\mu_{{\cal C}}$.
\item  Better: adopt exponentiated gradient descent \cite{Collins05,Collins08}
\item Derivation: summing on both sides of the update equation...
\begin{align*}
\mu^{(t+1)}_{icz} & = \sum_{y} {\bf 1\!}\left[y_c = z \right]  \alpha_{iy}^{(t)}
 \exp \left[  \lambda \langle w^*, \delta\psi_i(y) \rangle- \triangle(y_i,y)\right]   \\
& \propto \sum_{y} {\bf 1\!}\left[y_c = z \right]  \alpha_{iy}^{(t)}
 \exp \left[  \lambda \langle w^*, \psi_c(x_i,y_{iz}) - \psi_c(x_i,z) \rangle- \triangle(y_{iz},z)\right] 
 \\ & = \mu^{(t)}_{icz} \cdot
\exp \left[  \lambda \langle w^*, \psi_c(x_i,y_{iz}) - \psi_c(x_i,z) \rangle- \triangle(y_{iz},z)\right] 
\end{align*}
\begin{itemize}
\item $w^*$ can (representer theorem) computed from $\mu$ and $\psi_c$ (or via $k_c$), $\triangle_c$ terms. 
\end{itemize}
\item Similar for log-loss, faster convergence rates $O(\log 1/\epsilon)$. 
\end{itemize}
\end{frame}


%%%%%%%%%%%%%%%%%%%%%%%%%%%%%%%%%%%%%%%%%%%%%%%%%%
%%%%%%%%%%%%%%%%%%%%%%%%%%%%%%%%%%%%%%%%%%%%%%%%%%
%%%%%%%%%%%%%%%%%%%%%%%%%%%%%%%%%%%%%%%%%%%%%%%%%%
%
%\section{Spotlights: Extensions \& Improvements}
%\frame{\sectionpage}
%
%
%%%%%%%%%%%%%%%%%%%%%%%%%%%%%%%%%%%%%%%%%%%%%%%%%%%
%\begin{frame}
%\frametitle{Non-Convex Loss Bounds}
%
%\begin{itemize}
%\item Convex bounds (e.g.~Hinge loss) become very poor for large losses
%\item Use non-convex bound instead: ramp loss \cite{Collobert06,Chapelle08}
%\item Ramp loss: 
%\includegraphics[width=\textwidth]{graphics/ramploss.png}
%\begin{align*}
%l(x,y; f) 
%:= & \; \max_{y' \in \Ycal} \left[ \triangle(y,y') + f(x,y) - f(x,y')\right] \\
%&  -  \max_{y' \in \Ycal} \left[ f(x,y) - f(x,y') \right]
%\end{align*}
%\item Optimize with convex concave procedure (CCCP) \cite{Yuille03}
%\end{itemize}
%
%\end{frame}
%
%%%%%%%%%%%%%%%%%%%%%%%%%%%%%%%%%%%%%%%%%%%%%%%%%%
%%%%%%%%%%%%%%%%%%%%%%%%%%%%%%%%%%%%%%%%%%%%%%%%%%
%%%%%%%%%%%%%%%%%%%%%%%%%%%%%%%%%%%%%%%%%%%%%%%%%%

\section{Conclusion \& Discussion}
\frame{\sectionpage}


%%%%%%%%%%%%%%%%%%%%%%%%%%%%%%%%%%%%%%%%%%%%%%%%%%
\begin{frame}
\frametitle{Structured Prediction}

\begin{itemize}
\item Support Vector Machines: can be generalized to structured prediction in a scalable manner
\item Oracle-based architecture: decouples general learning method from domain-specific aspects  
\item Features \& loss function: can be incorporated in a flexible manner
\item Kernels: efficient dual methods exist that can rely on kernels (crossed feature maps, factor-level kernels)
\item Algorithms: rich set of scalable methods; cutting planes, subgradients, Frank-Wolfe, exponentiated gradient
\item Decomposition-based methods: can exploit insights and algorithms from approximate probabilistic inference
\item Conditional random fields: close relation (decomposition, dual, sparseness?) 
\item Applications: ever increasing number of applications and use cases
\end{itemize}
\end{frame}

%%%%%%%%%%%%%%%%%%%%%%%%%%%%%%%%%%%%%%%%%%%%%%%%%%
%%%%%%%%%%%%%%%%%%%%%%%%%%%%%%%%%%%%%%%%%%%%%%%%%%
%%%%%%%%%%%%%%%%%%%%%%%%%%%%%%%%%%%%%%%%%%%%%%%%%%

\small
\bibliographystyle {alpha}
\bibliography{thofmann}

\end{document}


