\documentclass{article}
\usepackage{amsfonts}
\usepackage{amsmath}
\usepackage{amssymb}

\DeclareMathOperator*{\argmax}{arg\,max}

\renewcommand{\th}{{\tilde \theta}}
\renewcommand{\Re}{{\mathbb R}}
\newcommand{\E}{{\bf E}}
\newcommand{\w}{{\bf w}}
\newcommand{\x}{{\bf x}}
\newcommand{\y}{{\bf y}}
\newcommand{\z}{{\bf z}}
\newcommand{\X}{{\bf X}}
\newcommand{\Y}{{\bf Y}}
\newcommand{\Z}{{\bf Z}}
\newcommand{\Xcal}{{\cal X}}
\renewcommand{\S}{{\cal S}}
\newcommand{\Ccal}{{\cal C}}
\newcommand{\Zcal}{{\cal Z}}
\newcommand{\tgamma}{{\stackrel{\gamma}{\longleftarrow}}}
\newcommand{\mycomment}[1]{}
\newcommand{\loglike}{{\mathcal L}}

\title{Stochastic Variational Inference}
\author{Thomas Hofmann, ETH Zurich}

\begin{document}

\maketitle 

\paragraph{Structure of Models} Global (hyper-)parameters $\alpha$, global parameters $\beta$, local variables $z_n$, observations $x_n$, 
\begin{align}
p(x,z,\beta|\alpha) = p(\beta|\alpha) \prod_n p(x_n,z_n|\beta) 
\end{align}

\paragraph{Exponential Family Assumptions} Further assumptions: probability of $\beta$ given everything else  as well as Probabilities of $z_{ni}$ given everything else are exponential families. This implies $p(x_n,z_n| \beta)$ as well as $p(\beta)$ are exponential.  Moreover it implies the conjugacy of the complete conditional of $\beta$ and its prior. 

\paragraph{ELBOW and Mean-field} 
\begin{align}
\log \int p(x,z,\beta) d \beta dz \geq \E_q\left[ \log p(x,z,\beta) \right] - \E_q \left[ \log q(z,\beta) \right]
\end{align}
with the choice of variational family 
\begin{align}
q(z,\beta) = q(\beta| \lambda) \prod_{n,j} q(z_{nj} | \phi_{nj})
\end{align}
Parameterizations are chosen by the form of the respective complete conditional distributions. 


%\paragraph{Multicausal Models} What about multicausal models?
%\begin{align}
%\log \frac{P(Z_d=1| Z_{-d}, X)}{P(Z_d=0| Z_{-d}, X)} + 
% = & \sum{e} (1-X_e) \theta_d  + \sum_{e} X_e \log\frac{1-\exp[ \theta_d + const]}{1- \exp[const]}
%\end{align}



\end{document} 
